\lecture{20}{Tuesday 19 February 2025}{02-19-25 Lecture}

Last time we showed that 
\begin{align}
	\sum_{n=1}^{\inf } \frac{1}{n^{2}}
\end{align}

converges. Now we will do a slightly different problem.

\begin{eg}
	$\sum_{n=1}^{\inf } \frac{1}{n}$
\end{eg}

The partial sums are 
\begin{align}
	S_m = 1 + \frac{1}{2} + \frac{1}{3} + \ldots + \frac{1}{m}
\end{align}

Observe that $\left( S_m \right) $ is an increasing sequence. To prove the statement, we will show that $\left( S_m \right) $ is \textit{not} bounded.

\begin{align}
	&S_4 = 1 + \frac{1}{2} + \left( \frac{1}{3} + \frac{1}{4} \right) > 1 + \frac{1}{2} + \left( \frac{1}{4} + \frac{1}{4}\right) \\
	&S_8\\
	&S_{16}\\
	&S_{32}\\
	&S_{2^{k}}
\end{align}

for $k \in \N$ we have

\begin{align}
	S_{2^{k}} = 1 + \frac{1}{2} + \left( \frac{1}{3} + \frac{1}{4} \right) + \left( \frac{1}{5} + \ldots + \frac{1}{8} \right)  + \ldots + \left( \frac{1}{2^{k-1} + 1} + \ldots + \frac{1}{2^{k}} \right) \\
	S_{2^{k}} > 1 + \frac{1}{2} + \left( \frac{1}{4} + \frac{1}{4} \right) + \left( \frac{1}{8} + \ldots + \frac{1}{8} \right) + \ldots + \left( \frac{1}{2^{k}} + \ldots + \frac{1}{2^{k}} \right)  \\
	= 1 + \frac{1}{2} + 2\left( \frac{1}{4} \right) + 4 \left( \frac{1}{8} \right) + \ldots + 2^{k-1} \frac{1}{2^{k}} \\
	= 1 + \frac{1}{2} + \frac{1}{2} + \frac{1}{2} + \ldots + \frac{1}{2} \\
	= 1 + k\left( \frac{1}{2} \right) \\
	\implies S_{2^{k}} > 1 + \frac{k}{2}
\end{align}

As the sequence $\left( 1 + \frac{k}{2} \right)_{k=1}^{\inf }$ is not bounded. Therefore $\left( S_m \right)_{m=1}^{\inf }$ is not bounded. Therefore $\left( S_m \right)_{m=1}^{\inf }$ is not convergent. Therefore $\sum_{n=1}^{\inf } \frac{1}{n}$ diverges.


\begin{theorem}
	The series
	\begin{align}
		\sum_{n=1}^{\inf } \frac{1}{n^{2}}
	\end{align}
	converges for $p > 1$ and diverges for $p \le 1$
\end{theorem}

\begin{proof}
	See textbook.
\end{proof}

Bolzano was a priest who first came up with the definition of the limit that we have been using. We will see some theorems named after him in this section.

\begin{definition}
	Let $\left( a_n \right) $ j be a sequence of real numbers and let $n_1 < n_2 < n_3 < \ldots$ be an increasing sequence of natural numbers, then the sequence 
	\begin{align}
	\left( a_{n_{1}}, a_{n_{2}}, a_{n_{3}}, \ldots \right) 
	\end{align}
	is called a \textbf{subsequence} of $\left( a_n \right)$  and is denoted by $\left( a_{n_{k}} \right) $ where $k \in \N$ indexes the subsequence.

\end{definition}

\begin{eg}
	Let $a_n = n^{2}$ i.e. 
	\begin{align}
		\left( a_n \right) = \left( 1, 4, 9, 16, 25, \ldots \right) 
	\end{align}
	Let $a_{n_{k}}$ = $\left( 2k \right)^{2} $ 
	\begin{align}
		\left( a_{n_{k}} \right) = \left( 4, 16, 36, \ldots \right) 
	\end{align}
	$\left( a_{n_{k}} \right) $ is a subsequence of $\left( a_n \right) $. Here $n_{k} = 2k$
	\begin{align}
		\left( 6^{2}, 11^{2}, 16^{2}, 21^{2}, \ldots \right) \text{ is also a subsequence} \\
		\left( 2^{2}, 2^{2}, 2^{2}, 3^{2}, 4^{2}, 5^{2}, \ldots \right) \text{ is \textit{not} a subsequence.}
	\end{align}
	The original sequence is 
	\begin{align}
		\left( 1, 2, 2, 2, 3, 4, 5, 6, \ldots \right) 
	\end{align}
	then 
	\begin{align}
		\left( 2, 2, 2, 3, 4, 5, 6, \ldots \right) 
	\end{align}
	is a subsequence.

\end{eg}

\begin{theorem}
	All sub sequences of a convergent sequence converge to the same limit as the original sequence.
\end{theorem}

\begin{proof}
	Assume that $\lim_{} a_n = a$ and let $\left( a_{n_{k}} \right) $ be a subsequence. Let $\e > 0$ be given, then $\exists N \in \N$ such that $\forall n \ge N$ we have 
	\begin{align}
		\left|a_n - a\right| < \e
	\end{align}
	For $k \ge N$ we observe that $n_{k} \ge k \ge N$. Therefore
	\begin{align}
		\left|a_{n_{k}} - a\right| < \e
	\end{align}
\end{proof}

\begin{note}
	The crucial thing to realize in the above is that $a_{n_{k}}$ is indexed by $k$ the $n_k$ is just there for emphasis.
\end{note}

\begin{eg}
	Let $0 < b < 1$, then $\lim_{} b^{n} = 0$
\end{eg}

\begin{proof}
	Observe that
	\begin{align}
		b > b^{2} > b^{3} > \ldots > 0
	\end{align}
	Therefore, $\left( b^{n} \right) $ is a decreasing sequence which is bounded. Then, by MCT, this sequence converges. Let $L \in \R$ such that $\lim_{n \to \inf } b_n = L$ Observe that for all $n \in \N\left( b \ge b^{n} \right) $. Therefore, by order limit theorem 
	\begin{align}
		1 > b \ge L
	\end{align}
	Similarly, $b^{n} \ge 0$ $\forall n \in \N$. Therefore $L \ge 0$. Therefore $0 \le L \le 1$ 
	Look at the subsequence $\left( b^{2n} \right) = \left( b^{2}, b^{4}, b^{6}, \ldots \right) $. Therefore, $\lim_{n \to \inf } b^{2n} = L$. Notice
	\begin{align}
		b^{n} b^{n} = b^{2n} \\
		a_n \cdot b_n
	\end{align}
	Therefore, by the ALT we have
	\begin{align}
		L \cdot L = L \\
		L^{2} = L \\
		L = 0 \text{ or } L = 1
	\end{align}
	But as $0 \le L < 1$. Therefore, $L = 0$ and $\lim_{n \to \inf } b^{n} = 0$
\end{proof}
