\lecture{9}{Friday 31 January 2025}{1-31-25 Lecture}

-Recall that a countably infinite set just means that $A \sim \N$.

-Recall that a countable set is either finite or countably infinite.

-Recall that an uncountable set is a set which is not countable. It is not clear a priori that these exist, but they do.

\begin{eg}
	$\N \sim \N^{2} \sim \Z \sim Z^{2}$ are all countable.
\end{eg}

\begin{theorem}
	The set $\Q$ is countable (ie it is a countable infinite set).
\end{theorem}

\begin{proof}
	Let $f: \N \to \Q$ be $f\left( n \right) = \frac{n}{2}$. This is an injective mapping. Every rational number $r \in \Q$ can be uniquely written as $\frac{p}{q} $ where $p \in \Z$, $q \in N$, and  $\text{ gcd}\left( p, q \right) = 1$. Define $g: \Q \to \Z^{2}$ given by $g\left( r \right) = \left( p, q \right) $. Clearly this is injective. As $\Z^{2} \sim \N$, $\exists h: \Z^{2} \to \N$ bijective. Thus $h \circ g: \Q \to \N$ is injective. Therefore, by the Cantor-Schroder-Bernstein theorem. $\Q \sim \N$, and $\Q$ is countable.
\end{proof}

\begin{theorem}
	$\R$ is uncountable.
\end{theorem}

\begin{proof}
	We proceed by contradiction. Therefore, $\N \sim \R$ i.e. $f: \N \to \R$ which is a bijection. Therefore, we can write $x_1 = f\left( 1 \right)$, $x_2 = f\left( 2 \right) $, $\ldots$. We have $\R = \{x_1, x_2, \ldots\}$. Consider a closed interval $I_1$ which does not contain $x_1$. Now let $I_2$ be a closed interval inside $I_1$ such that $x_2 \notin  I_2$. In general, given $I_n$ closed interval, construct a closed interval  $I_{n+1}$ such that
	\begin{enumerate}
		\item $I_{n+1} \subset I_n$.
		\item $x_{n+1} \notin I_{n+1}$.
	\end{enumerate}

	Consider the set $\cap_{n=1}^{\inf} I_n$. As $x_n \notin I_n \implies x_n \notin \cap_{n=1}^{\inf}$. As $f: \N \to \R$ is a bijection (As $\R = \{x_1, x_2, \ldots\}$) we have $\cap_{n=1}^{\inf} I_n = \emptyset$.
\end{proof}


But are there any infinities which are small than the cardinality of $\N$
\begin{theorem}
	If $B$ is a countable set and $A \subset B$, then $A$ is countable.	
\end{theorem}

\begin{proof}
	We will show the theorem by cases:
	\begin{itemize}
		\item $\left( B \text{ is a finite set}\right) $. As $A \subset B$, $A$ is also a finite set and $A $ is countable.
		\item ($B$ is countably infinite). If $A$ is finite, then obviously  $A$ is countable
		\item Now we assume that $A$ is an infinite set. As $B$ is countably infinite, we gave a bijection $f: \N \to B$. In particular, we can write
			\[
				B = \{f\left( 1 \right), f\left( 2 \right), \ldots\}
			.\]
			$A \subset B$ and is infinite. Let $n_1 = \text{ min} \{n \in \N : f\left( n \right) \in A\}$. More generally given $n_k$ we define $n_{k+1}$ as 
			 \[
				 n_{k+1} = \text{ min} \{n \in \N, n > n_k : f\left( n \right) \in A\}
			.\] 
			Define $g: \N \to A$ as $g\left( k \right) = f\left( n_k \right) $. By construction, $g$ is a bijection; therefore, $A \sim \N$ and $A$ is countable.
	\end{itemize}
\end{proof}





