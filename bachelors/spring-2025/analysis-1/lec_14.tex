\lecture{14}{Monday 10 February 2025}{02-10-25 Lecture}

\begin{theorem}
	The limit of a sequence, when it exists, is unique.
\end{theorem}

\begin{proof}
	Let $\left( a_n \right) $ be a sequence and assume that  $s, t \in \R$ such that $\lim_{} a_n = s$ and $\lim_{} a_n = t$. Let $\e > 0$ be arbitrary. As $\lim_{} a_n = s$, hence $\exists N_1 \in \N$ such that $\forall n \ge N_1$, we have $\left|a_n - s\right| < \frac{\e}{2}$. Similarly, as $\lim_{} a_n = t$, $\exists N_2 \in \N $ such that $\forall n \ge N_2$, we have $\left|a_n - t\right| < \frac{\e}{2}$. Let
	\begin{equation}
		N = \text{max}\{N_1, N_2\}
	\end{equation}
	hence $\left|a_N - s\right| < \frac{\e}{2}$ and $\left|a_N - t\right| < \frac{\e}{2}$.

	\begin{align}
		\left|s - t\right| = \left|\left( s - a_N \right) + \left( a_N - t \right) \right| \\
		\le \left|s - a_N\right| + \left|a_N - t\right|\\
		< \frac{\e}{2} + \frac{\e}{2} \\
		= \e \\
		\implies \left|s - t\right| < \e
	\end{align}
	As  $\e > 0$ is arbitrary, this implies that $s=t$
\end{proof}

\begin{definition}
	A sequence that does not converge is said to diverge.
\end{definition}

\begin{eg}
	Prove that the sequence $a_n = \left( -1 \right)^{n}$ diverges.
\end{eg}

\begin{note}
	The strategy for these is to assume that it converges, then show that it must converge to two different numbers.
\end{note}

\begin{proof}
	Suppose by contradiction, let $L \in \R$ be such that $\lim_{} a_n = L$. Therefore, given $\e = \frac{1}{2}$, there exists $N \in \N$ such that $\forall n \ge N$ we have $\left|a_n - L\right| < \frac{1}{2}$. Let $n_1 \ge N$ be odd $\implies \left|a_n - L\right| < \frac{1}{2} \implies \left|\left( -1 \right) ^{n_1} - L\right| < \frac{1}{2} \implies \left|\left( -1 \right) - L\right| < \frac{1}{2}$ as $n_1 + 1 \ge N$ and is even.
	\begin{equation}
		\implies \left|a_{n_1 + 1} - L\right| < \frac{1}{2} \implies \left|1 - L\right| < \frac{1}{2} \implies 2 < 1 
	\end{equation}
	a contradiction!
\end{proof}

\begin{eg}
	Prove that $\lim_{} \left( \frac{1}{n} \right) \neq 1$
\end{eg}

\begin{proof}
	By contradiction, assume that $\lim_{} \frac{1}{n} = 1$. Then for $\e = \frac{1}{2}$, $\exists N \in \N$ such that $\forall n \ge N$ 
	\begin{align}
		\left|\frac{1}{n} - 1\right| < \frac{1}{2} \\
		\implies 1 - \frac{1}{2} < \frac{1}{n} < 1 + \frac{1}{2} \\
		\implies \forall n \ge N \left( \frac{1}{2} < \frac{1}{n} \right)
	\end{align}

	By the Archimedean property, $\exists m \in \N$ such that
	\begin{align}
		m  \ge N \text{ and } \frac{1}{m} < \frac{1}{2} \\
	\end{align}
	This is a contradiction!
\end{proof}

\begin{definition}
	A sequence $\left( x_n \right) $ is \textbf{bounded} if there exists $M > 0$ such that $\left|x_n\right| \le M$ $\forall n \in \N$.
\end{definition}

\begin{theorem}
	Every convergent sequence is bounded.
\end{theorem}

This is a standard kind of argument that we will see again and again:

\begin{proof}
	Let $L \in \R$ be such that $\lim_{} x_n = L$. Hence for $\e = 1$, $\exists N \in \N$ such that $\forall n \ge N$ we have
	\begin{align}
		\left|x_n - L\right| < 1
	\end{align}
	Therefore, $\forall n \ge N$ 
	\begin{align}
		\left|x_n\right| &= \left|x_n - L + L\right| \\
				 &\le  \left|x_n - L\right|+ \left|L\right| \\
				 &< \left|L\right| + 1
	\end{align}
	Let $M = \text{max}\{\left|x_1\right|, \left|x_2\right|, \ldots, \left|x_{N-1}\right|, \left|L\right| + 1\} > 0$. We see that $\left|x_n\right| \le M$ $\forall n \in \N$. Hence $\left( x_n \right)$ is bounded.
\end{proof}












