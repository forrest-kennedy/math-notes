\lecture{18}{Monday 17 February 2025}{02-17-25 Lecture}

- DeLong Lecture today 3:30 - 4:30pm in Kitt Multipurpose room. Speaker: Prof. Laura DeMarco (Harvard)


\begin{definition}
	A sequence $\left( a_n \right) $ is called \textbf{increasing} if $a_n \le a_{n+1}$ for all $n \in N$. It is called \textbf{decreasing} if $a_n \ge a_{n+1}$ for all $n \in \N$. A sequence is called monotone if it is either increasing or decreasing.
\end{definition}

\begin{eg}
	$1, 1, 2, 2, 3, 3, 4, 4, \ldots$ is increasing.
\end{eg}

\begin{eg}
	$1, 1, 0, 0, -1, -1, \ldots$ is decreasing
\end{eg}

\begin{eg}
	$1, 1, 1, 1, 1, \ldots$	is constant and monotone.
\end{eg}

\begin{eg}
	$1, 0, 1, 0, 1, 0, 1, \ldots$ is \textit{not} monotone.
\end{eg}

\begin{theorem}[Monotone Convergence Theorem]
	If a sequence is monotone and bounded, then it converges.	
\end{theorem}
\begin{note}
	There are two enemies of convergence:
	\begin{enumerate}
		\item Oscillations (killed by monotone)
		\item Growth (killed by boundedness)
	\end{enumerate}
\end{note}

\begin{proof}
	Let $\left( a_n \right) $ be monotone and bounded. Let us assume that $\left( a_n \right) $ is increasing (the case for decreasing is proved similarly). Define the set
	\begin{align}
		S = \{a_n | n \in \N\} 
	\end{align}

	As $\left( a_n \right) $ is bounded, this means that the set $S$ is bounded above. Let $x = \sup S$ Now we just need to show that  $\lim_{} a_n = x$ to prove the statement. Let $\e > 0$ be given. As $x$ is the least upper bound, $x - \e$ is not an upper bound for $S$. Then there exists $n \in \N$ such that $x - \e < a_N$. Therefore, for all $n \ge N$ we have
	\begin{align}
		x- \e < a_N \le a_n \le x \\
		x - \e < a_n < x + \e \\
		\left|a_n - x\right| < \e
	\end{align}
	Hence proved.
\end{proof}

\begin{definition}
	Let $\left( b_n \right) $ be a sequence. An \textbf{infinite series} is a formal expression of the form 
	\begin{align}
		\sum_{n=1}^{\inf } b_n = b_1 +b_2 + b_3 + \ldots
	\end{align}
	We define the corresponding \textbf{sequence of partial sums}, $\left( S_m \right) $ by
	\begin{align}
		S_m = b_1 + b_2 + \ldots + b_m
	\end{align}
	we say that the series $\sum_{n=1}^{\inf } b_n$ \textbf{converges to B} if the sequence $\left( S_m \right) $ converges to $B$. In this case, we write  $\sum_{n=1}^{\inf } = B$.
\end{definition}

\begin{note}
	When we write the first sum, we are literally just writing symbols. If we want to assign meaning to this, we need to construct a sequence of partial sums $b_1, b_1 + b_2, b_1 +b_2 +b_3, \ldots$.
\end{note}

\begin{eg}
	Recall from day 1:
	\begin{align}
		b_n &= \left( -1 \right)^{n} \\
		S_1 &= b_1 = -1 \\
		S_2 &= b_1 + b_2 = 0 \\
		S_2 &= b_1 + b_2 + b_3 = -1 \\
		\ldots
	\end{align}
	Then construct the sequence: 
	\begin{align}
		\left( S_1, S_2, S_3, \ldots \right) = \left(-1, 0, -1, 0, -1, \ldots  \right) 
	\end{align}
	The sequence does not converge, therefore the series doesn't converge.
\end{eg}

\begin{eg}
	Consider 
	\begin{align}
		\sum_{n=1}^{\inf} \frac{1}{n^{2}} = \frac{1}{1} + \frac{1}{2^{2}} + \frac{1}{3^{2}} + \ldots
	\end{align}
	As all the terms in the series are positive, we observe that the sequence $\left( S_m \right) $ is an increasing sequence. Now we will apply a trick
	\begin{align}
		S_m &= 1 + \frac{1}{2^{2}} + \frac{1}{3^{2}} + \ldots + \frac{1}{m^{2}}\\
		&< 1 + \frac{1}{1 \cdot 2} + \frac{1}{2 \cdot 3} + \frac{1}{3 \cdot 4} + \ldots + \frac{1}{\left( m-1 \right)m} \\
		&= 1 + \left( \frac{1}{1} - \frac{1}{2} \right) + \left( \frac{1}{2} - \frac{1}{3} \right) + \left( \frac{1}{3} - \frac{1}{4} \right) + \ldots + \left( \frac{1}{m-1} - \frac{1}{m} \right) \\
		&= 1 + 1 - \frac{1}{m} \\
		&< 2
	\end{align}
	Therefore $S_m < 2$ for all $M \in \N$. Hence the sequence $\left( S_m \right) $ is bounded. As $\left( S_m \right) $ is an increasing bounded sequence, by the monotone convergence theorem, it converges.
\end{eg}

\begin{note}
	The above is the Basel Problem. The value that it converges to was found by Euler in 1734 and surprisingly is $\frac{\pi^{2}}{6}$. This is connected to the Riemann Zeta function. 
\end{note}
