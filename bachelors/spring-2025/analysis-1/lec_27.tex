\lecture{27}{Tuesday 05 March 2025}{03-05-25}

\section{Basic Topology of $\R$}

\begin{definition}
	Given $a \in \R$ and $\e > 0$, the $\e$-neighborhood of  $a$ is the set $V_{\e}\left( a \right) $ defined as
	\begin{align}
		V_{\e}\left( a \right) = \{x \in \R : \left|x - a\right| < \e\} 
	\end{align}
	i.e. $V_{\e}\left( a \right) = \left( a - \e, a + \e \right) $.
\end{definition}

\begin{definition}
	A set $O \subset  \R$ us called  \textbf{open} if for every $a \in O$, there exists $\e > 0$ such that $V_{\e}\left( a \right) \subset O$
\end{definition}

\begin{eg}
	$\left( 0, 1 \right) $ is an open set

	$[0, 1]$ is not an open set.

	$\R$ is an open set

	$\left( 0, 1 \right) \cup_{}^{} \left( 3, 4 \right) $ is an open set
	\begin{note}
		the union of open sets is open
	\end{note}
\end{eg}

\begin{theorem}
	We have
	\begin{enumerate}
		\item The union of an arbitrary collection of open sets is open.
		\item The intersection of a finite collection of open sets is open.
	\end{enumerate}
\end{theorem}

\begin{proof}
	\begin{enumerate}
		\item Let $\{O_{\lambda} : \lambda \in \Lambda\} $ be a collection of open sets. If $a \in \cup_{\lambda \in  \Lambda}^{} {} O_\lambda$, $\exists \alpha \in \Lambda$ such that $a \in O_{\alpha}$. Therefore as  $O_{\alpha}$ is open, $\exists  \e > 0$ such that $V_{\e}\left( a \right) \subset O_{\alpha}$. Therefore $V_{\e}\left( a \right) \subset \cup_{x \in \Lambda}^{}O_{\lambda}$. Hence $\cup_{\lambda \in  \Lambda}^{} O_{\lambda}$ is open.
		\item Let $\{O_1, O_2, O_3, \ldots, O_N\} $ be an open set. Let $a \in \cap_{i=1}^{N} O_i$. Therefore $a \in O_i$ for each $1 \le  i \le  N$. Then $\exists  \e_{i} > 0$ such that $V_{\e_{i}}\left( a \right) \subset O_i$. Let $\e = \text{ min}\{\e_{1}, \e_{2}, \ldots, \e_{N}\} > 0$. Therefore $V_{\e}\left( a \right) \subset O_i$ for all $1 \le i\le N$. Therefore $V_{\e}\left( a \right) \subset \cap_{i=1}^{N} O_i$

	\end{enumerate}	
\end{proof}



\begin{definition}
	Let $A \subset \R$. A point $x \in \R$ is called a \textbf{limit point of the set $A$}, if every $\e$-neighborhood, $V_{\e}\left( a \right) $ of $x$ intersects the set $A$ at some point other than $x$.
\end{definition}

\begin{note}
	Limit points are also called "cluster points" or "accumulation points" of a set.
\end{note}

\begin{eg}
	$A = \{1, \frac{1}{2}, \frac{1}{3}, \frac{1}{4}. \ldots\} $

	$x= 1$ is not a limit point because I can take an $\e$ small enough to contain no elements in $A$.

	$x= 0$ is a limit point of $A$


	$A = \left( 0, 1 \right) $ 

	Then if $0 \le x \le 1$ then $x$ is a limit point.
\end{eg}

\begin{eg}
	$\Q$. Every $x \in \R$ is a limit point of $\Q$.
\end{eg}

\begin{theorem}
	Let $A \subset \R$. A point $x \in \R$ is a limit point of $A$ if and only if there exists a sequence $\left( a_n \right) $ such that $a_n \in A$ and $a_n \neq x$ $\forall n \in \N$ and $\lim_{} a_n = x$
	\begin{itemize}
		\item ($\implies$ ) Let $x$ be a limit point of $A$. Consider the set $V_{\frac{1}{n}}\left( x \right) $. There exists $a_n \in V_{\frac{1}{n}}\left( x \right) \cap_{}^{} A$ such that $a_n \neq x$. Consider $\left( a_n \right) $. $\left|a_n - x\right| < \frac{1}{n}$, $a_n \in A$, $a_n \neq x$ $\forall n \in \N$. Given some $\e > 0$, let $N \in \N$ such that $\frac{1}{N} < \e \implies \forall n \ge N$, $\left|a_n - x\right| < \e$
		\item ($\impliedby$) Let $\left( a_n \right) $ be a sequence with $a_n \in A$, 
	\end{itemize}
\end{theorem}

