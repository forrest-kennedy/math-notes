\lecture{24}{Tuesday 26 February 2025}{02-26-25 Lecture}

The Cauchy Criterion. This is very important.

\begin{definition}
	A sequence $\left( a_n \right) $ is called a \textbf{Cauchy sequence} if for every $\e > 0$, $\exists N \in  \N$ such that whenever $m, n \ge N$ it follows that $\left|a_n - a_m\right| < \e$.
\end{definition}

In this lecture, we will build up to proving the following theorem:
\begin{theorem}
	A sequence is Cauchy if and only if it is convergent.
\end{theorem}

\begin{note}
	If a sequence converges, but we don't know how to find the value of the limit  $L$, then using the above, we can still show that the sequence converges, and still use the convergent limit theorems. This will take some buildup.
\end{note}

\begin{theorem}
	Every convergent sequence is a Cauchy sequence.
\end{theorem}

\begin{proof}
	Let $L \in  \R$ and let $\left( a_n \right) $ be a sequence with $\lim_{n \to \infty } a_n = L$. Let $\e > 0$ be given. Therefore $\exists N \in \N$ such that $\forall n\ge N$ we have 
	\begin{align}
		\left|a_n - L\right| < \frac{\e}{2}
	\end{align}
	If $m \ge N$ we also have
	\begin{align}
		\left|a_m - L \right| < \frac{\e}{2}
	\end{align}
	Therefore if $m,n \ge N$ we have
	\begin{align}
		\left|a_n -a_m\right| \le \left|a_n - L\right| + \left|L - a_m\right| \\
		< \frac{\e}{2} + \frac{\e}{2} \\
		= \e
	\end{align}
\end{proof}

\begin{theorem}
	Cauchy sequences are bounded.
\end{theorem}

\begin{theorem}
	Let $\e = 1$ and let $N \in \N$ such that $\forall m,n \ge N$ we have 
	\begin{align}
		\left|a_m - a_n\right| < 1
	\end{align}
	Let $m = N$. Therefore for all $n \ge N$ we have 
	\begin{align}
		\left|a_N - a_n\right| < 1 \\
		\implies \left|a_n\right| \le \left|a_n - a_N\right| + \left|a_N\right| < \left|a_N\right| + 1
	\end{align}
	Let $M = \text{max}\{\left|a_1\right|, \left|a_2\right|, \left|a_3\right|, \ldots, \left|a_{N-1}\right|, \left|a_N\right| + 1\} $. Then we see that 
	\begin{align}
		\left|a_n\right| \le M
	\end{align}
	for all $n \in \N$. Hence $\left( a_n \right)$ is bounded.
\end{theorem}

\begin{theorem}
	A sequence converges if and only if it is Cauchy.
\end{theorem}

\begin{proof}
	Let $\left( x_n \right) $ be a Cauchy sequence. Therefore, $\left( x_n \right) $ is a bounded sequence. By Bolzano-Weirstrauss theorem, there exists a convergent subsequence $\left( x_{n_{k}} \right) $. Let 
	\begin{align}
		x = \lim_{k \to \infty} x_{n_{k}}
	\end{align}
	Now notice that what we want to claim is that $\lim_{n \to \infty}x_n = x$. So let $\e > 0$ be given. $\exists N \in \N$ such that $\forall m,n \ge N$ we have
	\begin{align}
		\left|x_n - x_n\right| < \frac{\e}{2}
	\end{align}
	There also exist $k \in \N$ such that $\forall k \ge K$ with $K \ge N$ such that $\forall n, m \ge N$ we have 
	\begin{align}
		\left|x_{n_{k}} - x\right| < \frac{\e}{2}
	\end{align}
	Observe that $n_{k} \ge K \ge  N$. Hence 
	\begin{align}
		\left|x_{n_{k} - x}\right| < \frac{\e}{2}
	\end{align}
	and hence $\forall n \ge K$ 
	\begin{align}
		\left|x_n -x\right| \le \left|x_{n} - x_{n_{k}}\right| + \left|x_{n_{k}} - x\right|\\
		< \frac{\e}{2} + \frac{\e}{2} \\
		= \e
	\end{align}
	Therefore
	\begin{align}
		\left|x_n - x\right| < \e
	\end{align}
	For all $n \ge K$. Hence $\lim_{} x_n = x$
\end{proof}

Now starting back with series. . . 

Properties of Infinite Series

\begin{align}
	\sum_{n=1}^{\infty} b_n = b_1 + b_2 + b_3 + \ldots
\end{align}

\begin{note}
	Remember the distinction between sequence and series..
\end{note}

How do we tell if a series if convergent? Construct a sequence of partial sums $\left( s_m \right) $ 
\begin{align}
	s_m - b_1 + b_2 + \ldots + b_m
\end{align}

If $\left( s_m \right) $ converges, the we say that the series $\sum_{n=1}^{\infty} b_n$ converges.

If $\lim_{} s_m = L$, the $\sum_{n=1}^{\infty} b_n = L$

\begin{theorem}[Algebraic Limit Theorem for Series]
	If $\sum_{n=1}^{\infty} a_n = A$ and $\sum_{n=1}^{\infty} b_n = B$, then
	\begin{enumerate}
		\item $\sum_{n=1}^{\infty} ca_n = cA$ for any $c \in \R$
		\item $\sum_{n=1}^{\infty} \left( a_n + b_n \right) = A + B$
	\end{enumerate}
\end{theorem}

\begin{note}
	The other two don't work!!!
\end{note}

\begin{proof}
	Let $\left( s_m \right) $ and $\left( t_m \right) $ be the sequence of partial sums i.e.
	\begin{align}
		s_m = a_1 + a_2 + \ldots + a_m \\
		t_m = b_1 + b_2 + \ldots + b_m
	\end{align}
	Therefore $\lim_{m \to \infty} s_m = A$ and $\lim_{m \to \infty} t_m = B$. By the Algebraic Limit Theorem
	\begin{align}
		\lim_{m \to \infty} c s_m = cA \\
		\implies \sum_{n=1}^{\infty} c a_n = cA
	\end{align}
	\begin{note}
		$c a_1 + c a_2 + \ldots + c a_m = c s_n$
	\end{note}
	Again by ALT
	\begin{align}
		\sum_{n=1}^{\infty} \left( a_n + b_n \right) = A + B
	\end{align}
	\begin{note}
		$\left( a_1 + b_1 \right) + \left( a_2 + b_2 \right) + \ldots + \left( a_m + b_m \right) = s_m + t_m$.
	\end{note}
\end{proof}




