\lecture{6}{Monday 27 January 2025}{Homework 2}


1. Compute, without proof, the supremum and infimum (if they exist) of each of the following sets:

2. Let $A \subset \R$ and $B \subset \R$ be two non-empty sets, each of which is bounded above. If $s=\text{sup }A $ and $t=\text{sup }B $, find and prove a formula for $\text{sup } A \cup B$

\begin{proof}
	We argue that $\sup A \cup B = max\left( s, t \right) $ by cases:
	\begin{enumerate}
		\item ($s>t$) WLOG with respect to the $s < t$ case, since $t=\sup B$, we have  $t \ge b$ for all $b \in B$, thus by using our case assumption we get $s > t \ge b$ for all $b \in b$ and $s \ge  a$ for all $a \in A$ by the definition of the supremum of a set. Therefore, $s \ge u$ for all $u \in A \cup B$ and $\sup A \cup B = s = max\left( s, t \right) $.
		\item ($s = t$) If $s=t$ by the definition of the supremum we have $s \ge a$ for all $a \in A$ and $t = s \ge b$ for all $b \in B$. Thus  $max\left( s, t \right) = s = t \ge a$ for all $a \in A$ and $max\left( s, t \right) = s = t \ge b$ for all $b \in B$. Therefore, $\sup A \cup B = max\left( s, t \right) $.
	\end{enumerate}
\end{proof}

3. Let $A \subset \R$ and $B \subset \R$ be two non-empty sets, each of which is bounded above. 
\begin{enumerate}
	\item If $\sup A < \sup B$, show that there exists $b \in B$ such that $b$ is an upper bound for $A$.
	\item Given an example to show that this is not always the case if we only assume $\sup A \le \sup B$.
\end{enumerate}

\begin{proof}
	\begin{enumerate}
		\item Combining the definition of the supremum of a set and the given, we get
			\[
				a \le \sup A < \sup B 
			.\] 
			Thus $\sup B$ is an upper bound for $A$. Since $\sup B > \sup A$, it must be that $\sup B$ is not the least upper bound of $A$. Then by negating the $\e$ Characterization of the Supremum we see that there must exist an $\e > 0 $ such that $\sup B - \e > a$ for all $a \in A$. We also know from the given that for all $\e > 0 $, there exists an $b \in B$ such that $\sup B - \e < b$  
	\end{enumerate}
\end{proof}

5. Let $A \subset \R$ and $c \in \R$. We define the set $cA$ as:
\[
	cA = \{ca | a \in A\}
.\] 

If $A$ is non-empty and bounded above and $c\ge 0$, then prove that $sup \text{ } cA$ = $c \cdot sup \text{ } A $.

\begin{proof}
	By the definition of the supremum, we have $\sup cA \ge ca \implies \frac{1}{c} \sup cA \ge a$. Then $\frac{1}{c} \sup cA$ is an upper bound for $A$. But $\sup A$ is the \textit{least} upper bound of $A$, so it must be that $\frac{1}{c} \sup cA\ge \sup A$, thus $\sup cA \ge c\sup A$.

	By the definition of the supremum, we have $\sup A \ge a$ for all $a \in A \implies c \sup A \ge ca$ $\forall a \in A$. Thus $c \sup A$ is an upper bound of  $cA$. But $\sup cA$ is the \textit{least} upper bound of $cA$, so it must be that $c \sup A \ge \sup cA$.

	Thus, $\sup cA \ge c \sup A$ and $\sup cA \le c \sup A$; therefore, $\sup cA = c \sup A$. 
\end{proof}






