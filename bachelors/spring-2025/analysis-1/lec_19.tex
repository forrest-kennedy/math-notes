\lecture{19}{Monday 17 February 2025}{Homework 5}

1. Let $\left( a_n \right) \to 0$. Use the Algebraic limit theorem to compute each of the following limits (assuming the functions are always defined). Justify all of your actions.

\begin{enumerate}
	\item $\lim_{} \left( \frac{1 + 2a_n}{1 + 3 a_n - 4 a_{n}^{2}} \right) $ 
	\item $\lim_{} \left( \frac{\left( a_n + 2 \right)^{2} - 4}{a_n} \right) $
	\item $\lim_{} \left( \frac{\frac{2}{a_n} + 3}{\frac{1}{a_n} + 5} \right) $
\end{enumerate}
%----------------------------------------------------------------------------------
2. Prove that the following sequences diverge:
\begin{enumerate}
	\item The sequence $\left( a_n \right) $ where 
		\begin{align}
			a_n = \left( -1 \right)^{n} n^{2} + 1
		\end{align}
	\item The sequence $\left( a_n \right)$ where
		\begin{align}
			\left( -1 \right)^{n} + \frac{1}{n}
		\end{align}
	\end{enumerate}
%----------------------------------------------------------------------------------

3. (Squeeze Theorem). Show that if $x_n \le y_n \le z_n$ for all $n \in \N$ and if $\lim_{} x_n = \lim_{} z_n = l$, then $\lim_{} y_n = l$ as well.
%----------------------------------------------------------------------------------

4. Let $x_n \ge 0$ for all $n \in \N$.
\begin{enumerate}
	\item If  $\left( x_n \right) \to 0$, show that $\sqrt{s_n}  \to 0$.
	\item If $\left( x_n \right) \to x$, show that $\left( \sqrt{x_n}  \right) \to \sqrt{x} $.
\end{enumerate}
%----------------------------------------------------------------------------------

5. Consider the sequence $\left( b_n \right) $ where $b_n = \sqrt{n^{2} + 2n} - n$. Prove that $\left( b_n \right) $ is convergent and find its limit.
%----------------------------------------------------------------------------------

6. Give an example of each of the following:
\begin{enumerate}
	\item Sequences $\left( a_n \right)$ and $ \left( b_n \right) $, which both diverge, but whose sum $\left( a_n + b_n \right) $ converges.
	\item Sequences $\left( a_n \right) $ and $\left( b_n \right)$, which both diverge, but whose product $\left( a_n b_n \right) $ converges.
	\item Convergent sequences $\left( a_n \right) $ and $\left( b_n \right) $ with $a_n < b_n$ for all $n \in \N$ such that $\lim_{} \left( a_n \right) = \lim_{} \left( b_n \right) $.
	\item A convergent sequence $\left( b_n \right) $ with $b_n \neq 0$ for all $n \in \N$, such that $\left( 1 / b_n \right)$ diverges.
	\item Two sequence $\left( a_n \right) $ and $\left( b_n \right) $ so that $\left( a_n \right) $ is unbounded, $\left( b_n \right) $ is bounded, and $\left( a_nb_n \right)$ converges.
	\item Two sequences $\left( a_n \right)$ and $ \left( b_n \right) $, where $\left( a_n b_n \right) $ and $\left( a_n \right) $ converge but $\left( b_n \right) $ does not.
\end{enumerate}
%----------------------------------------------------------------------------------

7. Let $\left( a_n \right) $ be a bounded (not necessarily convergent) sequence, and assume that $\lim_{} b_n = 0$. Show that $\lim_{a_n b_n} = 0$. Why are we not allowed to use the Algebraic limit theorem to prove this?









