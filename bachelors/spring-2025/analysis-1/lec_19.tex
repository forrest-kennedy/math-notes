\lecture{19}{Monday 17 February 2025}{Homework 5}

1. Let $\left( a_n \right) \to 0$. Use the Algebraic limit theorem to compute each of the following limits (assuming the functions are always defined). Justify all of your actions.

\begin{enumerate}
	\item $\lim_{} \left( \frac{1 + 2a_n}{1 + 3 a_n - 4 a_{n}^{2}} \right) $ 
	\begin{align}
		&= \frac{\lim_{} \left( 1 + 2 a_n \right)}{\lim_{} \left( 1 + 3 a_n - 4a_n^{2} \right)} \\
		&= \frac{\lim_{} 1 + \lim_{} 2 \cdot \lim_{} a_n}{\lim_{} 1 + \lim_{} 3 \cdot  \lim_{} a_n - \lim_{} 4 \cdot \lim_{} a_n \cdot \lim_{} a_n} \\
		&= 1
	\end{align}

	\item $\lim_{} \left( \frac{\left( a_n + 2 \right)^{2} - 4}{a_n} \right) $
		\begin{align}
			&= \frac{\lim_{} \left( a_n + 2 \right) \cdot \lim_{} \left( a_n + 2 \right) - \lim_{} 4}{\lim_{} a_n} \\
			&= \frac{\left( \lim_{} a_n + \lim_{} 2 \right) \left( \lim_{} a_n + \lim_{} 2 \right) - \lim_{} 4}{\lim_{} a_n} \\
			&= \frac{\left( \lim_{} a_n + 2 \right) \left( \lim_{} a_n + 2 \right) - 4}{\lim_{} a_n} \\
			&= \frac{\left( \lim_{} a_n \right)^{2} + 4 \lim_{} a_n + 4 - 4}{\lim_{} a_n} \\
			&= \lim_{} a_n + 4 = 4
		\end{align}

	\item $\lim_{} \left( \frac{\frac{2}{a_n} + 3}{\frac{1}{a_n} + 5} \right)$
		\begin{align}
			&= \frac{\lim_{} \left( \frac{2}{a_n} + 3 \right) }{\lim_{} \left( \frac{1}{a_n} + 5 \right)} \\
			&= \frac{\lim_{} \frac{2}{a_n} + \lim_{} 3}{\lim_{} \frac{1}{a_n} + \lim_{} 5} \\
			&= \frac{\frac{\lim_{} 2}{\lim_{} a_n} + 3}{\frac{\lim_{} 1}{\lim_{} a_n} + 5} \cdot \frac{\lim_{} a_n}{\lim_{} a_n} \\
			&= \frac{\lim_{} 2 + 3 \lim_{} a_n}{\lim_{} 1 + 5 \lim_{} a_n} = 2
		\end{align}
\end{enumerate}
%----------------------------------------------------------------------------------
2. Prove that the following sequences diverge:
\begin{enumerate}
	\item The sequence $\left( a_n \right) $ where 
		\begin{align}
			a_n = \left( -1 \right)^{n} n^{2} + 1
		\end{align}

		


	\begin{proof}
			Assume for the sake of contradiction that $\left( a_n \right) $ converges. Since $\left( a_n \right) $ converges, it is bounded. Therefore $\exists M > 0$ such that 
			\begin{align}
				&\forall n \in \N\left( \left|a_n\right| \le M \right) \\	
				&\forall n \in \N \left( \left|\left( -1 \right)^{n} n^{2} + 1 \right| \le M\right) 
			\end{align}
			If we force $n$ to be even, then
			\begin{align}
				\forall n \in \N \left( n^{2} + 1 \le M \right) 
			\end{align}
			But by the Archimedean principle, we can always pick $N$ such that $N$ is even and $N > \sqrt{M - 1} $. Then we have
			\begin{align}
				N > \sqrt{M - 1} \implies N^{2} + 1 > M 	
			\end{align}
			Which contradicts our assumption that $\left( a_n \right) $ converges. Thus $\left( a_n \right) $ must diverge.

	\end{proof}






	\item The sequence $\left( a_n \right)$ where
		\begin{align}
			a_n = \left( -1 \right)^{n} + \frac{1}{n}
		\end{align}



	\begin{proof}
			Assume for the sake of contradiction that $\left( a_n \right) $ converges. Let $a = \lim_{} a_n$. Then pick  $\e < a$; and, by the definition of convergence, we have:
			\begin{align}
				\left|a_n - a\right| &\le \e\\	
				\left|\left( -1 \right)^{n} + \frac{1}{n} - a \right| &\le \e \\
				-\e \le \left( -1 \right)^{n} + \frac{1}{n} - a &\le \e 	\\
				a -\e \le \left( -1 \right)^{n} + \frac{1}{n} &\le \e+ a 	
			\end{align}
			If we force $n$ to be even and consider the left side of the inequality, then for all even $n$ we have
			\begin{align}
				a - \e \le \frac{1}{n}
			\end{align}
			But notice that because of how we picked $\e$, we know that $a - \e > 0$. Then, by the Archimedean property of $\N$, there exists an $N \in \N$ such that $N$ is even and 
			\begin{align}
			a - \e > \frac{1}{N}. 
			\end{align}
			Thus, we have reached a contradiction. Therefore, the sequence must diverge.
	\end{proof}



	
	\end{enumerate}
%----------------------------------------------------------------------------------

3. (Squeeze Theorem). Show that if $x_n \le y_n \le z_n$ for all $n \in \N$ and if $\lim_{} x_n = \lim_{} z_n = l$, then $\lim_{} y_n = l$ as well.

\begin{proof}
	Take the first given statement, and subtract $l$ from everything: 
	\begin{align}
		x_n - l \le y_n - l \le z_n - l
	\end{align}
	Since we are given that $\lim_{} x_n = \lim_{} z_n = l$, we can say that for all $\e > 0$ there exists $n_1, n_2 \in \N$ such that, if $N_1 \ge  n_1$ and $N_2 \ge n_2$  
	\begin{align}
		\left|x_{N_1} - l\right| < \e \implies -\e < x_{N_1} - l < \e \\
		\left|z_{N_2} - l\right| < \e \implies -\e < z_{N_2} - l < \e
	\end{align}
	Therefore, if we let $p = \text{max }\{n_1, n_2\} $ we can say that for all $P > p$
	\begin{align}
		-\e < x_P - l \le y_P -l \le z_P - l < \e  
	\end{align}
	Thus for all $\e > 0$ there exists a $P \in \N$ such that 
	\begin{align}
		\left|y_P - l\right| < \e 
	\end{align}
	Therefore, $\lim_{} y_n = l$
	
\end{proof}
%----------------------------------------------------------------------------------

4. Let $x_n \ge 0$ for all $n \in \N$.
\begin{enumerate}
	\item If  $\left( x_n \right) \to 0$, show that $\sqrt{x_n}  \to 0$.
	\begin{proof}
		We know that if $\lim_{} x_n = 0$ that for all $\e > 0$ there exists an $n \in \N$ such that if $N > n$, $\left|x_n - 0\right| < \e$. Since $x_n \ge 0$ for all $n \in \N$, we have for all $\e > 0$ there exists an $n$ such that $x_n < \e$ $\implies \sqrt{x_n} < \e \implies \sqrt{x_n} - 0 < \e $. Now since $x_n \ge 0$ we know that $\sqrt{x_n} > 0$. Since we also have  $\e > 0$, we can just take an absolute value on both sides to get 
	\end{proof}


	\item If $\left( x_n \right) \to x$, show that $\left( \sqrt{x_n}  \right) \to \sqrt{x} $.
\end{enumerate}
%----------------------------------------------------------------------------------

5. Consider the sequence $\left( b_n \right) $ where $b_n = \sqrt{n^{2} + 2n} - n$. Prove that $\left( b_n \right) $ is convergent and find its limit.
%----------------------------------------------------------------------------------

6. Give an example of each of the following:
\begin{enumerate}
	\item Sequences $\left( a_n \right)$ and $ \left( b_n \right) $, which both diverge, but whose sum $\left( a_n + b_n \right) $ converges.

	Consider $\left( a_n \right) = n$ and $\left( b_n \right) = -n$

	\item Sequences $\left( a_n \right) $ and $\left( b_n \right)$, which both diverge, but whose product $\left( a_n b_n \right) $ converges.

	Consider $\left( a_n \right) = \left( -1 \right)^{n}$ and $\left( b_n \right) = \left( -1 \right)^{n+2}$
	\item Convergent sequences $\left( a_n \right) $ and $\left( b_n \right) $ with $a_n < b_n$ for all $n \in \N$ such that $\lim_{} \left( a_n \right) = \lim_{} \left( b_n \right) $.

	Consider $\left( a_n \right) = \left( \frac{1}{4} \right)^{n}$ and $\left( b_n \right) = \left( \frac{1}{2} \right)^{n}$
	\item A convergent sequence $\left( b_n \right) $ with $b_n \neq 0$ for all $n \in \N$, such that $\left( 1 / b_n \right)$ diverges.

	Consider $\left( b_n \right) = \frac{1}{n}$
	\item Two sequence $\left( a_n \right) $ and $\left( b_n \right) $ so that $\left( a_n \right) $ is unbounded, $\left( b_n \right) $ is bounded, and $\left( a_nb_n \right)$ converges.

	Consider $\left( a_n \right) = n$ and $\left( b_n \right) = \frac{1}{n}$
	\item Two sequences $\left( a_n \right)$ and $ \left( b_n \right) $, where $\left( a_n b_n \right) $ and $\left( a_n \right) $ converge but $\left( b_n \right) $ does not.

	Consider $\left( a_n \right) = \frac{1}{n}$ and $\left( b_n \right) = n$
\end{enumerate}
%----------------------------------------------------------------------------------

7. Let $\left( a_n \right) $ be a bounded (not necessarily convergent) sequence, and assume that $\lim_{} b_n = 0$. Show that $\lim_{} a_n b_n = 0$. Why are we not allowed to use the Algebraic limit theorem to prove this?

\begin{proof}
	From the triangle inequality, we have
	\begin{align}
		\left|a_n b_n - 0\right| = \left|a_n\right| \left|b_n\right| = \left|a_n\right| \left|b_n - 0\right|
	\end{align}
	Since $a_n$ is bounded, we know that there exists an $M$ such that $a_n \le M$ for all $n \in \N$. Therefore
	\begin{align}
		\left|a_n b_n - 0\right| = \left|a_n\right|\left|b_n - 0\right| < M \left|b_n - 0\right|
	\end{align}

	Finally, we know that for all $\e > 0$,  $\left|b_n - 0\right| < \e$, so, from the definition of convergence, $\left|b_n - 0\right| < \frac{\e}{M}$ and
	\begin{align}
		\left|a_n b_n - 0\right| = \left|a_n\right|\left|b_n - 0\right| < M \left|b_n - 0\right| < M \frac{\e}{M} = \e
	\end{align}

	Therefore, $\lim_{} a_n b_n = 0$. Notice that we could not use the Algebraic limit theorem because that theorem requires both $a_n$ and $b_n$ to converge. We know that $b_n$ converges, but we are only given that $a_n$ is bounded, so it is not necessarily convergent.
\end{proof}









