\lecture{22}{Friday 21 February 2025}{Homework 6}

1. Give an example of each of the following:
\begin{alphabetize}
	\item A monotone sequence that diverges.

		$\left( x_n \right) = n$
	\item A sequence that has no convergent subsequence.

		$\left( x_n \right) = n$
	\item A divergent sequence that has a convergent subsequence.

		We need a sequence which is bounded, but diverges: 

		Consider $\left( x_n \right)  = \left( -1 \right)^{n}$ and notice that $\left( x_{n_k} \right) $ converges when $n_k = 2k$.
	\item A sequence that does not contain $0$ or $1$ as a term, but contains subsequences converging to each of these values.

		$\left( x_n \right)_{n=2}^{\infty } = 1 / n$
	\item A sequence containing subsequences converging to every point in $\R$

		We need a sequence which visits every point in $\R$ infinitely many times so that we can construct our sequence. Take
		\begin{equation}
			\left( x_n \right) = \tan\left( n \right) 
		\end{equation}
		Now if we define the subsequence $\left( x_{n_k} \right) $ where $n_k = \frac{1}{k} + \tan^{-1}\left( \alpha \right) + 2 \pi k$. Then we have that $\lim_{} x_{n_k} = \alpha$ for all $\alpha \in  \R$.
\end{alphabetize}
%------------------------------------------------------------------------
2. Consider the sequence $\left( a_n \right) $ defined as follows: Let $a_1 = \sqrt{2} $ and for $n \ge 1$ let 
	\begin{align}
		a_{n+1} = \sqrt{2 + a_n} 
	\end{align}
Hence the sequence is
	\begin{align}
		\left( \sqrt{2}, \sqrt{2 + \sqrt{2}}, \sqrt{2 + \sqrt{2 + \sqrt{2} } }, \ldots \right) 
	\end{align}
Prove that $\left( a_n \right) $ converges and find its limit. (Hint: First using induction prove that $1 < a_n < 2$ for all $n \in \N$. Then show that $\left( a_n \right) $ is a monotonic bounded sequences and then find its limit)

\begin{proof}
	We will show that $\left( a_n \right) $ converges by showing that it is monotonic and bounded. We will show each inductively starting with monotonicity:
	\begin{itemize}
		\item (Base Case) when $n=1$ we have $\sqrt{2} \overset{?}{\le} \sqrt{2 + \sqrt{2}} $ which is true.
		\item (Inductive Step) Assume that $a_{n+1} \ge a_n$, then $\sqrt{2 + a_{n+1}} \ge \sqrt{2 + a_{n}} \implies a_{n+2} \ge a_{n+1}$. 
	\end{itemize}
	Now we will inductively prove that the sequence is bounded by $2$:
	\begin{itemize}
		\item (Base Case) when $n = 1$, we have  $1 \overset{?}{\le } \sqrt{2} \overset{?}{\le } 2$ which is true.
		\item (Inductive Step) Assume that $1 < a_n < 2$, then $3 < a_n + 2 < 4$ and $\sqrt{3} < \sqrt{2 + a_n} < 2$. Therefore $a_{n+1} < 2$.
	\end{itemize}
	We have shown that $\left( a_n \right) $ is both monotone and bounded, therefore it converges.	
\end{proof}

\begin{proof}
	To show that $\lim_{} \left( a_n \right) = 2$, we use a little trick. Consider
	\begin{align}
		a_{n+1} = \sqrt{2 + a_n}.
	\end{align}
	Since we already proved that $\lim_{} \left( a_n \right) $ exists, we are justified in saying that:
	\begin{align}
		\lim_{} \left( a_{n+1} \right) = \lim_{} \left( \sqrt{2 + a_n}\right).
	\end{align}
	Notice that $\lim_{} \left( a_{n+1} \right) = \lim_{} \left( a_n  \right) $, so let $l = \lim_{} \left( a_{n+1} \right) = \lim_{} \left( a_n  \right) $. Then we have
	\begin{align}
		&l = \sqrt{2 + l} \\
		\implies &l^{2} - l - 2 = 0
	\end{align}
	Thus, since we already proved that $a_n > \sqrt{3} $ for all $n \in \N$, we know that $l = \lim_{} \left( a_n \right) = 2$.
\end{proof}
%------------------------------------------------------------------------
3. (AM-GM inequality) Prove the Arithmetic-Geometric mean inequality: For $x, y \ge 0$ show that $\frac{x+y}{2} \ge \sqrt{xy} $.

\begin{proof}
	We are given that $x, y \ge 0$, so
	\begin{align}
		x + y &\ge  0 \\
		\frac{x+y}{2} &\ge 0\\
		\frac{x+y}{2} + \sqrt{xy} - \sqrt{xy}  &\ge 0\\	
		\frac{x+y}{2} + \sqrt{xy} &\ge  \sqrt{xy}\\	
		\left( \frac{x+y}{2} + \sqrt{xy}\right) \left( \frac{x+y}{2} - \sqrt{xy}  \right) &\ge  \sqrt{xy} \left( \frac{x+y}{2} - \sqrt{xy}  \right) \\		
		\frac{\left( x+y \right)^{2}}{4} - xy &\ge \sqrt{xy} \left( \frac{x+y}{2} \right) - xy \\ 
		\left( x+y \right) \left( x+y \right)  &\ge 2 \sqrt{xy} \left( x + y \right) 
	\end{align}
	By examining the final equation, we see that it must be that $\left( x + y \right) \ge 2 \sqrt{xy} $ so that the inequality holds. Hence proved.
\end{proof}

%------------------------------------------------------------------------
4. (Calculating square roots) Let $a_1 = 2 $ and for $n \ge 1$ define
	\begin{align}
		a_{n+1} = \frac{1}{2} \left( a_n + \frac{2}{a_n} \right) 
	\end{align}
Prove that $\left( a_n \right) $ converges and then show that $\lim_{}\left( a_n \right) = \sqrt{2} $. (Hint: First using induction and the AM-GM inequality prove that $\sqrt{2} \le a_n \le 2$ for all $n \in  \N$. Then show that $\left( a_n \right) $ is a monotonic bounded sequence and then find its limit)

\begin{remark}
	This gives us an algorithm to compute $\sqrt{2} $. This can be generalized to compute $\sqrt{x} $ and more generally $\sqrt[n]{x}$ for any $x > 0$. 
\end{remark}

\begin{proof}
	First we will prove that the sequence converges by showing that it is monotonic and bounded inductively. We will start with monotonicity:
	\begin{itemize}
		\item (Base Case) When $n=1$ we have $a_1 = 2$ and $a_2 = 3 / 2$. $3 / 2 > 2$ holds.
		\item (Inductive Step) Assume that the statement holds for $a_n$, then 
			 \begin{align}
				a_n \le a_{n+1}\\
				\frac{1}{2}\left( a_n + \frac{2}{a_n} \right)  \le \frac{1}{2} \left(a_{n+1} + \frac{2}{a_n}\right) \le \frac{1}{2}\left( a_{n+1} + \frac{2}{a_{n+1}} \right) \\	
				a_{n+1} \le a_{n+2}
			\end{align}
	\end{itemize}
	Now we will show boundedness using the AM-GM inequality which we proved in problem $3$. Consider:
	\begin{align}
		a_{n+1} = \frac{1}{2} \left( a_n + \frac{2}{a_n} \right)
	\end{align}
	Now we will show boundedness using the AM-GM inequality which we proved in problem $3$. Consider $a_{n+1} = \frac{1}{2}\left( a_n \right) +\frac{2}{a_n}$, then we know by AM-GM Ineqiality that:
	\begin{align}
		a_{n+1} = \frac{1}{2}\left( a_n + \frac{2}{a_n} \right) \ge   \sqrt{a_n \cdot \frac{2}{a_n}} = \sqrt{2} 
	\end{align}

	$\left( a_n \right) $ is monotonic and bounded, therefore it converges.
\end{proof}

\begin{proof}
	We will use the same trick to find this limit. Since we know that the limit exists and is non-zero, we can feel confident in writing $l = \lim_{} \left( a_n \right) = \lim_{} a_{n+1}$, then
	\begin{align}
		\lim_{} a_{n+1} = \frac{1}{2}\left( \lim_{} a_n + \frac{2}{\lim_{} a_n}\right) \\
		\implies l = \frac{1}{2}\left( l + \frac{2}{l} \right) 
	\end{align}
	Solving the above gives $l = \lim_{} \left( a_n \right)  = \sqrt{2} $

\end{proof}
%------------------------------------------------------------------------
5. Consider a sequence $\left( a_n \right)_{n=1}^{\inf }$ defined by 
\begin{align}
	a_n = \frac{1}{n} + \frac{1}{n+1} + \frac{1}{n+2} + \ldots + \frac{1}{2n}
\end{align}
Prove that $\left( a_n \right)_{n=1}^{\inf }$ is a convergent sequence. (Hint: prove that the sequence is bounded and monotonic).

%------------------------------------------------------------------------
6. Let $\left( a_n \right) $ be a bounded sequence and let $a \in \R$. Show that if every converges subsequence of $\left( a_n \right) $ converges to $a$, then the sequence $\left( a_n \right) $ itself converges and $\lim_{} a_n = a$

\begin{proof}
	If every $\left( a_{n_{k}} \right) $ converges to $a$, then the "trivial subsequence" where $n_{k} = k$ also converges to $a$. Now, assume for the sake of contradiction that $\lim_{}\left( a_n \right) \neq a$, then $\exists \e > 0$ such that 
	 \begin{align}
		\left|a_n - a\right| > \e 
	\end{align}
	But that implies that there exists  a subsequence of $\left( a_n \right) $ which does not converge to $a$, i.e. the trivial subsequence, which contradicts our assumption that all subsequences of $\left( a_n \right) $ converge to $a$.
\end{proof}
%------------------------------------------------------------------------
7. Consider the series 
\begin{align}
	\sum_{n=1}^{\inf } \frac{1}{n\left( n+1 \right) }
\end{align}
Prove that this series converges and find the limit. (See the proof of convergence of $\sum_{n=1}^{\inf } \frac{1}{n^{2}}$ done in class)

%------------------------------------------------------------------------
8. (Bonus problem. Will not be asked in the quiz or exams) Suppose the sequence $\left( a_n \right)_{n=1}^{\inf }$ converges to $a \in \R$. Prove that the sequence $\left( c_N \right)_{N=1}^{\inf }$ defined as $c_N = \frac{1}{N} \sum_{n=1}^{N} a_n$, satisfies $\lim_{N \to \inf } c_N = a$

