\lecture{22}{Friday 21 February 2025}{Homework 6}

1. Give an example of each of the following:
\begin{alphabetize}
	\item A monotone sequence that diverges.
	\item A sequence that has no convergent subsequence.
	\item A divergent sequence that has a convergent subsequence.
	\item A sequence that does not contain $0$ or $1$ as a term, but contains subsequences converging to each of these values.
	\item A sequence containing subsequences converging to every point in $\R$
\end{alphabetize}
%------------------------------------------------------------------------
2. Consider the sequence $\left( a_n \right) $ defined as follows: Let $a_1 = \sqrt{2} $ and for $n \ge 1$ let 
	\begin{align}
		a_{n+1} = \sqrt{2 + a_n} 
	\end{align}
Hence the sequence is
	\begin{align}
		\left( \sqrt{2}, \sqrt{2 + \sqrt{2}}, \sqrt{2 + \sqrt{2 + \sqrt{2} } }, \ldots \right) 
	\end{align}
Prove that $\left( a_n \right) $ converges and find its limit. (Hint: First using induction prove that $1 < a_n < 2$ for all $n \in \N$. Then show that $\left( a_n \right) $ is a monotonic bounded sequences and then find its limit)

%------------------------------------------------------------------------
3. (AM-GM inequality) Prove the Arithmetic-Geometric mean inequality: For $x, y \ge 0$ show that $\frac{x+y}{2} \ge \sqrt{xy} $.

%------------------------------------------------------------------------
4. (Calculating square roots) Let $a_1 = 2 $ and for $n \ge 1$ define
	\begin{align}
		a_{n+1} = \frac{1}{2} \left( a_n + \frac{2}{a_n} \right) 
	\end{align}
Prove that $\left( a_n \right) $ converges and then show that $\lim_{}\left( a_n \right) = \sqrt{2} $. (Hint: First using induction and the AM-GM inequality prove that $\sqrt{2} \le a_n \le 2$ for all $n \in  \N$. Then show that $\left( a_n \right) $ is a monotonic bounded sequence and then find its limit)

\begin{remark}
	This gives us an algorithm to compute $\sqrt{2} $. This can be generalized to compute $\sqrt{x} $ and more generally $\sqrt[n]{x}$ for any $x > 0$. 
\end{remark}


%------------------------------------------------------------------------
5. Consider a sequence $\left( a_n \right)_{n=1}^{}$ 
%------------------------------------------------------------------------
6.
%------------------------------------------------------------------------
7.
%------------------------------------------------------------------------
8.

