\lecture{22}{Friday 21 February 2025}{Homework 6}

1. Give an example of each of the following:
\begin{alphabetize}
	\item A monotone sequence that diverges.

		$\left( x_n \right) = n$
	\item A sequence that has no convergent subsequence.

		$\left( x_n \right) = n$
	\item A divergent sequence that has a convergent subsequence.

		We need a sequence which is bounded, but diverges: 

		Consider $\left( x_n \right)  = \left( -1 \right)^{n}$ and notice that $\left( x_{n_k} \right) $ converges when $n_k = 2k$.
	\item A sequence that does not contain $0$ or $1$ as a term, but contains subsequences converging to each of these values.

		$\left( x_n \right)_{n=2}^{\infty } = 1 / n$
	\item A sequence containing subsequences converging to every point in $\R$

		We need a sequence which visits every point in $\R$ infinitely many times so that we can construct our sequence. Take
		\begin{equation}
			\left( x_n \right) = \tan\left( n \right) 
		\end{equation}
		Now if we define the subsequence $\left( x_{n_k} \right) $ where $n_k = \frac{1}{k} + \tan^{-1}\left( \alpha \right) + 2 \pi k$. Then we have that $\lim_{} x_{n_k} = \alpha$ for all $\alpha \in  \R$.
\end{alphabetize}
%------------------------------------------------------------------------
2. Consider the sequence $\left( a_n \right) $ defined as follows: Let $a_1 = \sqrt{2} $ and for $n \ge 1$ let 
	\begin{align}
		a_{n+1} = \sqrt{2 + a_n} 
	\end{align}
Hence the sequence is
	\begin{align}
		\left( \sqrt{2}, \sqrt{2 + \sqrt{2}}, \sqrt{2 + \sqrt{2 + \sqrt{2} } }, \ldots \right) 
	\end{align}
Prove that $\left( a_n \right) $ converges and find its limit. (Hint: First using induction prove that $1 < a_n < 2$ for all $n \in \N$. Then show that $\left( a_n \right) $ is a monotonic bounded sequences and then find its limit)

%------------------------------------------------------------------------
3. (AM-GM inequality) Prove the Arithmetic-Geometric mean inequality: For $x, y \ge 0$ show that $\frac{x+y}{2} \ge \sqrt{xy} $.

%------------------------------------------------------------------------
4. (Calculating square roots) Let $a_1 = 2 $ and for $n \ge 1$ define
	\begin{align}
		a_{n+1} = \frac{1}{2} \left( a_n + \frac{2}{a_n} \right) 
	\end{align}
Prove that $\left( a_n \right) $ converges and then show that $\lim_{}\left( a_n \right) = \sqrt{2} $. (Hint: First using induction and the AM-GM inequality prove that $\sqrt{2} \le a_n \le 2$ for all $n \in  \N$. Then show that $\left( a_n \right) $ is a monotonic bounded sequence and then find its limit)

\begin{remark}
	This gives us an algorithm to compute $\sqrt{2} $. This can be generalized to compute $\sqrt{x} $ and more generally $\sqrt[n]{x}$ for any $x > 0$. 
\end{remark}


%------------------------------------------------------------------------
5. Consider a sequence $\left( a_n \right)_{n=1}^{\inf }$ defined by 
\begin{align}
	a_n = \frac{1}{n} + \frac{1}{n+1} + \frac{1}{n+2} + \ldots + \frac{1}{2n}
\end{align}
Prove that $\left( a_n \right)_{n=1}^{\inf }$ is a convergent sequence. (Hint: prove that the sequence is bounded and monotonic).

%------------------------------------------------------------------------
6. Let $\left( a_n \right) $ be a bounded sequence and let $a \in \R$. Show that is every converges subsequence of $\left( a_n \right) $ converges to $a$, then the sequence $\left( a_n \right) $ itself converges and $\lim_{} a_n = a$

%------------------------------------------------------------------------
7. Consider the series 
\begin{align}
	\sum_{n=1}^{\inf } \frac{1}{n\left( n+1 \right) }
\end{align}
Prove that this series converges and find the limit. (See the proof of convergence of $\sum_{n=1}^{\inf } \frac{1}{n^{2}}$ done in class)

%------------------------------------------------------------------------
8. (Bonus problem. Will not be asked in the quiz or exams) Suppose the sequence $\left( a_n \right)_{n=1}^{\inf }$ converges to $a \in \R$. Prove that the sequence $\left( c_N \right)_{N=1}^{\inf }$ defined as $c_N = \frac{1}{N} \sum_{n=1}^{N} a_n$, satisfies $\lim_{N \to \inf } c_N = a$

