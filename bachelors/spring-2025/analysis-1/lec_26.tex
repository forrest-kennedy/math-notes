\lecture{26}{Monday 03 March 2025}{03-03-25 Lecture}

\begin{theorem}[Absolute converges test]
	If the series $\sum_{n=1}^{\infty} a_n$ converges, then $\sum_{n=1}^{\infty} \left|a_n\right|$ also converges.
\end{theorem}

\begin{proof}
	It is enough to show that $\sum_{}^{} a_n$ satisfies the Cauchy criterion. Let $\e > 0$ be given. Hence $\exists N \in \N$ such that whenever $n > m \ge N$ we have
	\begin{align}
		\left|\left|a_{m+1}\right| + \left|a_{m+2}\right| + \ldots + \left|a_n\right|\right| < \e \\
		\implies \left|a_{m+1} + a_{m+2} + \ldots + a_n\right| < \e
	\end{align}
	Hence proved.
\end{proof}

\begin{eg}
	\begin{align}
		\sum_{n=1}^{\infty} \frac{\sin\left( n \right) }{n^{2}}
	\end{align}
	Observe that 
	\begin{align}
		\frac{\sin\left( n \right) }{n^{2}} \le \frac{1}{n^{2}}
	\end{align}
	As $\sum_{}^{} \frac{1}{n^{2}}$ converges, by the comparison test $\sum_{}^{} \left|\frac{\sin\left( n \right) }{n^{2}}\right|$ converges. Therefore by the absolute convergence test $\sum_{}^{} \frac{\sin\left( n \right) }{n^{2}}$ also converges.
\end{eg}

\begin{definition}
	Consider the series $\sum_{}^{} a_n$
	\begin{enumerate}
		\item If $\sum_{}^{} \left|a_n\right|$ converges, then we say that the series $\sum_{}^{} a_n$ converges absolutely.
		\item If the series $\sum_{}^{} a_n$ converges, but $\sum_{}^{} \left|a_n\right|$ diverges, then we say that $\sum_{}^{} a_n$ converges conditionally.
	\end{enumerate}
\end{definition}

\begin{eg}
	\begin{align}
		\sum_{}^{} \frac{\sin\left( n \right) }{n^{2}}
	\end{align}
	By previous logic, the series converges absolutely.
\end{eg}

\begin{eg}
	\begin{align}
		\sum_{}^{} \frac{\left( -1 \right)^{n}}{n}
	\end{align}
	converges by the alternating series test but $\sum_{}^{} \frac{1}{n}$ diverges. Therefore $\sum_{}^{} \frac{\left( -1 \right)^{n}}{n}$
\end{eg}

\begin{note}
	The conditional convergence is the most painful.
\end{note}

\begin{theorem}[Alternating series test]
	Let $\left( a_n \right) $ be a sequence satisfying
	\begin{enumerate}
		\item $a_1 \ge a_2 \ge a_3 \ge \ldots$ and $a_n \ge 0$ $\forall n \in \N$
		\item $\lim_{} a_n = 0$
	\end{enumerate}
	Then $\sum_{n=1}^{\infty} \left( -1 \right)^{n+1} a_n$ converges.
\end{theorem}

\begin{proof}
	Consider the partial sum $s_m$ 
	\begin{align}
		s_m = a_1 + a_2 + \ldots + \left( -1 \right)^{m+1} a_m
	\end{align}
	Consider the subsequence $\left( s_{2n} \right)_{n=1}^{\infty} = \left( s_2, s_4, s_6, \ldots \right) $
	\begin{align}
		s_2 = a_1 - a_2 \\
		s_4 = a_1 -a_2 + a_3 - a_4 = a_1 - \left( a_2 - a_3 \right) - a_4 \le a_1
	\end{align}
	Therefore
	\begin{align}
		s_{2n+2} - s_{2n} = a_{2n+1} - a_{2n+2} \ge 0
	\end{align}
	Therefore $\left( s_{2n} \right)_{n=1}^{\infty}$ is an increasing sequence
	\begin{align}
		s_{2n} = a_1 - a_2 + a_3 \ldots - a_{2n-2} + a_{2n-1} - a_{2n} \\
		= a_1 - \left( a_2 - a_3 \right) - \ldots - \left( a_{2n-2} - a_{2n-1} \right) - a_{2n} \\
		\le a_1
	\end{align}
	$\left( s_{2n} \right) $ is an increasing sequence which is bounded above and hence by MCT it converges. Let $L \in \R$ such that 
	\begin{align}
		\lim_{n \to \infty} s_{2n} = L
	\end{align}
	Now I claim that $\lim_{} s_n = L$. To show this, let $\e > 0$ be given. $\exists N_1 \in \N$ such that $\forall  n \ge N_1$ we have 
	\begin{align}
		\left|s_{2n} - L \right| < \frac{\e}{2}
	\end{align}
	As $\lim_{} a_n = 0$, $\exists N_2 \in \N$ such that $\forall n \ge N_2$ 
	\begin{align}
		\left|a_n\right| < \frac{\e}{2}
	\end{align}
	Let $N = \text{max}\{N_2, N_1\} $. Hence $\forall n \ge N$ we have 
	\begin{itemize}
		\item (Case 1: $n$ is even) 
			\begin{align}
				\left|s_n - L\right| < \frac{\e}{2} < \e
			\end{align}
		\item (Case 2: $n$ is odd) 
			\begin{align}
				\left|s_n - L\right| &\le \left|s_n - s_{n+1}\right| + \left|s_{n+1} - L\right|\\
						     &\le \left|a_{n+1}\right| + \frac{\e}{2} \\
						     &< \frac{\e}{2} + \frac{\e}{2} \\
						     &= \e
			\end{align}
	\end{itemize}
\end{proof}


\begin{eg}
	\begin{align}
		a_n = \frac{1}{n}
	\end{align}
	By the AST $\sum_{}^{} \frac{\left( -1 \right)^{n+1}}{n} = 1 - \frac{1}{2} + \frac{1}{3} - \frac{1}{4} \ldots$ converges.
\end{eg}

\begin{theorem}[Ratio Test]
	Consider the series $\sum_{n=1}^{\infty} a_n$ where $a_n \neq 0$ $\forall n \in \N$. Assume that $\exists r$ such that $0 \le r < 1$ and
	\begin{align}
		\lim_{n \to \infty} \left|\frac{a_{n+1}}{a_n}\right| = r
	\end{align}
	Then the series $\sum_{n=1}^{\infty} a_n$ converges absolutely.
\end{theorem}

\begin{proof}
	Given in homework.
\end{proof}

\begin{eg}
	For some $x \in \R$
	\begin{align}
		\sum_{n= 0}^{\infty} \frac{x^{n}}{n!}
	\end{align}
	If $x=0$, then
	\begin{align}
		1 + 0 + 0 + 0 + \ldots
	\end{align}
	If $x \neq 0$, the $a_n = \frac{x^{n}}{n!}$ is non-zero.
	\begin{align}
		\left|\frac{a_n+1}{a_n}\right| = \left| \frac{\frac{x^{n+1}}{\left( n+1 \right)!}}{\frac{x^{n}}{n!}} \right| = \left|\frac{x}{n+1}\right|
	\end{align}
	By ALT $\lim_{x \to \infty} \left|\frac{x}{n+1}\right| = 0 = r$. Therefore, by the ration test, this series converges absolutely $\forall x \in \R$.
\end{eg}














