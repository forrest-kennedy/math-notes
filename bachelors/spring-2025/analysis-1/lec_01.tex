\lecture{1}{Tuesday 28 January 2025}{Complete Analysis Theorems List}



\section{The Real Numbers}
\begin{definition}[Definition of a "function"]
	Given sets $A, B$ a function of $A \to B$ is a mapping that takes ea    ch element of $A$ to a single element of $B$.
\end{definition}

\begin{definition}[Definition of the "absolute value function"]
	The \textbf{absolute value function"} is defined as $\left|\cdot\right|: \R \to \R$ such that:
	\[
		\left|x\right| =
		\begin{cases} 
		      x & x \ge  0 \\
		      -x & x < 0 \\
   		\end{cases}
	.\] 
\end{definition}

\begin{theorem}[The Triangle Inequality]
	With respect to multiplication and division, the absolute value function satisfies:
	\begin{enumerate}
		\item $\left|ab\right| = \left|a\right|\left|b\right|$
		\item $\left|a+b\right| \le \left|a\right| + \left|b\right|$
	\end{enumerate}
\end{theorem}

\begin{proof}
	We will show the theorem by cases WLOG:
	\begin{enumerate}
		\item $\left( a = 0 \right) $ $\left|a + b\right| = \left|0 + b\right| = \left| b\right| = \left|0\right| + \left|b\right| = \left|a\right| + \left|b\right|$
		\item $\left(a > 0, b > 0 \right)$ By the definition of the absolute value function we have  $\left|a + b\right| = a + b = \left|a\right| + \left|b\right|$
		\item $\left( a < 0, b < 0 \right)$ By the definition of the absolute value function we have $\left|a + b\right| = -\left( a+b \right) = -a + \left( -b \right) = \left|a\right| + \left|b\right|$
		\item $\left( a > 0, b < 0 \right) $ By the definition of the absolute value, we have  $\left|a\right| = a$ and $\left|b\right| = -b$, so $\left|a\right| + \left|b\right| = a + \left( -b \right) $. We want to show that $\left|a\right| + \left|b\right| = a + \left( -b \right) \ge \left|a + b\right|$, so again we consider all the possible cases:
			\begin{enumerate}
				\item $\left( a + b = 0 \right)$ We have $a + \left( -b \right) \overset{?}{\geq} \left|0\right| = 0$. Indeed, since $a>0$ and $b<0$ we have $a > b$, and our equality holds.
				\item $\left( a + b > 0 \right) $ We have $a + \left( -b \right) \overset{?}{\geq}a+b$. Since $b < 0$, we have $-b > 0$. Comparing the LHS and RHS the equality holds.
				\item $\left( a + b < 0 \right) $ We have $a + \left( -b \right)  \overset{?}{\geq} -a + \left( -b \right) $. Comparing the LHS and the RHS, the equality holds.
			\end{enumerate}
	\end{enumerate}

	The above considerations exhaust all possible choices for $a$ and $b$. In all cases. we see that $\left|a+b\right| \le \left|a\right| +\left|b\right|$
\end{proof}


\begin{theorem}[The $\e$ criteria for equality]
	Two real numbers $a$ and $b$ are equal if and only if for every real number $\eps > 0$ it follows that $\left|a - b\right| < \eps$.
\end{theorem}

\begin{proof}
	We will show the theorem in both directions:
	\begin{itemize}
		\item $\left( \implies \right) $ Given $a = b$, we have $a - b = 0 < \eps$ for all $\eps > 0$.
		\item $\left( \impliedby \right)$ Assume that for every $\eps > 0$, $\left|a-b\right| < \e$ and, FSOC, that $a \neq b$. Then, let $\e_0 = a-b$ which we know is nonzero because $a \neq b$. Now, $\left|a-b\right| = \e_0$ and $\left|a-b\right| < \e_0$ by our first assumption. We have reached a contradiction, therefore the reverse implication must hold.
	\end{itemize}	
\end{proof}

\begin{definition}[Bounded Above Property of Subsets of $\R$]
       A set $A \subset \R$ is \textbf{bounded above} if there exists a number $b \in \R$ such that $a \le b$ $\forall a \in A$. The number $b$ is called an \textbf{upper bound} for $A$.
 \end{definition}
 
  
 \begin{definition}[Bounded Below Property of Subsets of $\R$]
         A set $A \subset \R$ is \textbf{bounded below} if there exists a number $b \in \R$ such that $b \le a$ $\forall a \in A$. The number $b$ is called a \textbf{lower bound} for $A$.
 \end{definition}
 
 \begin{definition}[The Least Upper Bound]
         An element $s \in \R$ is called the \textbf{least upper bound} for $A \subset \R$ if $s$ meets two conditions:
         \begin{enumerate}
                 \item $s$ is an upper bound for $A$
                 \item $\forall b$ where $b$ is an upper bound, $s \le b$.
         \end{enumerate}
 \end{definition}
 
 \begin{definition}[The Greatest Lower Bound]
         An element $l \in \R$ is called the \textbf{greatest lower bound} for $A \subset \R$ if $l$ meets two conditions:
         \begin{enumerate}
                 \item $l$ is a lower bound for $A$
 		\item $\forall b$ where $b$ is an upper bound, $l \ge  b$.
\end{enumerate}
\end{definition}


\begin{definition}[The Maximum is a Set]
	A real number $a_0$ is a \textbf{maximum} of the set $A$ if $a_0$ is an elemnt of $A$ and $a_0 \ge a$ for all $a \in A$. Similarly, a number $a_1$ is a \textbf{minimum} of $A$ if $a_1 \in A$ and $a_1 \le a$ for all $a \in A$.
\end{definition}


\begin{theorem}[The $\e$  Characterization of the Supremum]
	Assume $s \in R$ is an upper bound for a set $A \subset \R$. Then, $s = \sup A$ if and only if, for every choice of $\e > 0$, there exists an element $a \in A$ satisfying $s - \e < a$.
\end{theorem}

\begin{proof}
       	We will show that both the implication and the inverse implication are true:
	
	\begin{itemize}
		\item $\left( \implies \right) $If $s$ is the \textit{least} upper bound of $A$, then $s - \e$ is not an upper bound for $A $, thus there exists an  $a \in A$ such that $s - \e < a$.
		\item $\left( \impliedby \right) $ Assume $s$ is an upper bound of $A$ and that for every $\e > 0$, $s-\e <a$. That is, no number smaller than $s$ is an upper bound of $A$. Thus for all $b$ where $b$ is an upper bound of $A$, $s \le b$. Since we assumed that $s$ is an upper bound, $s$ meets both conditions to be the supremum. 
	\end{itemize}	
\end{proof}

\begin{axiom}[Axoim of Completeness]
	Every nonempty set of real numbers that is bounded above has a least upper bound.	
\end{axiom}

\begin{theorem}[Nested Interval Property of Subsets of $\R$]
	For each $n \in \N$, assume we are given a closed interval $I_n = [a_n, b_n] = \{x \in R : a_n \le x\le b_n\}$. Assume also that each $I_n$ contains $I_{n+1}$. Then, the resulting nested sequence of closed intervals
	\[
	I_1 \supset I_2 \supset I_3 \supset \ldots
	.\] 
	has a nonempty intersection; that is, $\cap_{n=1}^{\inf} I_n \neq \emptyset$.
\end{theorem}

\begin{proof}
	Let $A = \{a_n : n \in \N\}$ and $B = \{b_n : n \in \N\}$, then let $\alpha = \sup A$. From the definition of the supremum, we have  $\alpha \ge a_n$ for all $n \in \N$. Because of how we defined our sets, every  $b_n$ is an upper bound of $A$, so we have $\alpha \le b_n$ for all $n \in \N$. Thus $a_n \le \alpha \le b_n$ and $\alpha \in I_n$. Therefore,  $I_n$ is nonempty.
\end{proof}

\begin{theorem}[Archimedean Property]
	The theorem has two parts:
	\begin{enumerate}
		\item Given any number $x \in \R$, there exists an $n \in \N$ satisfying $n > x$.
		\item Given any real number  $y>0$, there exists an $n \in \N$ satisfying $\frac{1}{n} < y$.
	\end{enumerate}
\end{theorem}

\begin{proof}
	Statement 1 in the above theorem is equivalent to the statement: $\N$ is not bounded above. FSOC, assume that $\N$ is bounded above, then let $\alpha = \sup \N$. By the definition of the supremum, $\alpha - 1$ is not an upper bound. Thus, $\alpha - 1 < n$ for some $n \in \N$ implies $\alpha < n + 1$, but $n+1 \in \N$ by definition so $\alpha$ is less than some natural number and cannot be the supremum, a contradiction! Thus $\N$ is not bounded above, and we have proven statement 1. To prove statement 2, let $x = \frac{1}{y}$ and substitute into the expression in statement 1.
\end{proof}

\begin{definition}[Sequence]
	A \textbf{sequence} is a function whose domain is  $\N$.
\end{definition}

\begin{definition}[Convergent Property of a Sequence / Limit of a Sequence]
	A sequence $a_n$ \textbf{converges} to a real number $a$, if for every $\e > 0$ there exists $N \in \N$ such that whenever $n \ge N$, we have $\left|a_n - a\right| < \e$. In this case we write
	\begin{align}
		\lim_{n \to \inf} a_n = \lim_{} a_n = a
	\end{align} 
\end{definition}

\begin{definition}[$\e$-Neighborhood]
	Given a real number $a \in \R$ and a positive number $\e > 0$, the set
	\begin{align}
		V_{\e} \left( a \right)  = \{x \in \R : \left|x - a\right| < \e\} 
	\end{align}
	is called the \textbf{$\e$-neighborhood} of $a$. 
\end{definition}

\begin{definition}[Topological Definition of the Convergent Property/Limit of a Sequence]
	A sequence $\left( a_n \right) $ converges to $a$ if every $\e$-neighborhood of $a$ contains all but a finite number of the terms of $\left( a_n \right) $.
\end{definition}

\begin{theorem}[Limit Uniqueness Theorem]
	The limit of a sequence, when it exists, is unique.
\end{theorem}

\begin{proof}
	For the sake of contradiction, let $\left( a_n \right) $ be a sequences which converges to both $s$ and $t$. Then we know that $\exists N_1, N_2 \in \N$ such that for all $\e > 0$ 
	\begin{align}
		\left|a_{N_{1}} - s\right| < \frac{\e}{2} \text{   and   } \left|a_{N_{2}} - t\right| < \frac{\e}{2}
	\end{align}
	Now let 
	\begin{align}
		N = \text{ max}\{N_{1}, N_{2}\}.
	\end{align}
	And consider $\left|s - t\right|$ We can now use the "adding zero" algebraic trick and the triangle inequality:
	\begin{align}
		\left|s - t\right| &= \left|\left( s - a_N \right) + \left( a_N - t \right) \right| \\
				   &\le  \left|a_N - s\right| + \left|a_N - t\right| \\
				   &< \frac{\e}{2} + \frac{\e}{2} \\
				   &= \e
	\end{align}
	Thus $\left|s - t\right| < \e$ for all $\e > 0$, by the $\e$-Criteria for Equality, $s=t$.
\end{proof}

\begin{definition}[Divergent Property of a Sequence]
	A sequence that does not converge is said to  \textbf{diverge}.
\end{definition}

\begin{definition}[Bounded Property of a Sequence]
	A sequence $\left( x_n \right) $ is \textbf{bounded} if there exists a number $M > 0$ such that $\left|x_n\right| \le M$ for all $n \in \N$.
\end{definition}

\begin{theorem}[Convergence-Boundedness Theorem]
	Every convergent sequence is bounded.
\end{theorem}

\begin{proof}
	Assume that the sequence $\left( x_n \right) $ converges to $l$. Then we can say that for some particular $\e$, say $\e = 1$, that $\exists N \in  \N$ such that if $n \ge N$, then $x_n \in \left( l - 1, l + 1 \right) $. We don't know for sure if $l$ is positive or negative, but we can say for sure that
	\begin{align}
		\left|x_n\right| < \left|l\right| + 1 		
	\end{align}
	For all $n \ge N$. Therefore if we let 
	\begin{align}
		M = \text{max}\{\left|x_1\right|, \left|x_2\right|, \left|x_3\right|, \ldots, \left|x_{N-1}\right|, \left|l\right| + 1\} 
	\end{align}
	it follows that $\left|x_n\right| \le M$ for all $n \in \N$ as desired. 
\end{proof}

\begin{theorem}[Partial Sequence Gap Theorem]
        Suppose $\left( a_n \right) $ is a convergent sequence with $\lim_{} a_n = L$. If $L\neq 0$ and $a_n\neq 0$ for all $n \in \N$, then $\exists \delta > 0$ such that $\left|a_n\right| \ge \delta > 0$ for all $n \in \N$.
\end{theorem}

\begin{proof}
        As $L \neq 0$, choose $\e = \frac{\left|L\right|}{2} > 0$ $\exists N \in \N$ such that $\forall n \ge N$ we have
        \begin{align}
                \left|a_n - L\right| < \frac{\left|L\right|}{2}\\
        \end{align}
        for $n \ge N$ we have
        \begin{align}
                \left|L\right| = \left|L - a_n + a_n \right|\le \left|L - a_n\right| + \left|a_n\right| \le \frac{\left|L\right|}{2} + \left|a_n\right|\\
        \end{align}
        Therefore, for all $n \ge N$ we have
        \begin{align}
                \frac{\left|L\right|}{2} \le \left|a_n\right| \\
        \end{align}
Define $\delta = \text{ min}\{\left|a_1\right|, \left|a_2\right|, \ldots, \left|a_{N-1}\right|, \left|\frac{L}{2}\right|\} > 0$. We see that  $\left|a_n\right| \ge \delta > 0$ $\forall n \in \N$.
\end{proof}


\begin{theorem}[Algebraic Limit Theorem]
	Let $\lim_{} a_n = a$, and $\lim_{} b_n =b$. Then,
	\begin{alphabetize}
		\item $\lim_{} ca_n = ca$ for all $c \in \R$;
		\item $\lim_{} a_n + b_n = a + b$;
		\item $\lim_{} a_n b_n = ab$;
		\item $\lim_{} \left( a_n / b_n \right) = a / b$, provided $b \neq 0$.
	\end{alphabetize}
\end{theorem}

\begin{proof}
	We will prove each in turn:
	\begin{alphabetize}
		\item Consider
			\begin{align}
				\left|ca_n - ca\right| \\
				= \left|c\right|\left|a_n - a\right|
			\end{align}
			We know from the given that $\exists N \in  \N$ such that if $n \ge N$ we have $\left|a_n - a\right| < \e$ therefore
			\begin{align}
				\left|ca_n - ca\right| = \left|c\right|\left|a_n - a\right| < \left|c\right| \frac{\e}{\left|c\right|} = \e
			\end{align}
		\item Consider
			\begin{align}
				\left|\left( a_n + b_n \right) - \left( a - b \right) \right| \\
				= \left|\left( a_n - a \right) + \left( b_n - b \right) \right|
			\end{align}
			Apply the triangle inequality
			\begin{align}
				\le \left|a_n - a\right| + \left|b_n - b\right| 
			\end{align}
			Finally apply the fact that $a_n$ and $b_n$ converge to get
			\begin{align}
				\left|\left( a_n + b_n \right) - \left( a - b \right) \right| \le \frac{\e}{2} + \frac{\e}{2} = \e
			\end{align}
		\item Consider 
			\begin{align}
				\left|a_n b_n - ab\right|
			\end{align}
			Use the add-subtract trick and the triangle inequality:
			\begin{align}
				\left|a_n b_n - ab\right| &= \left|a_n b_n - ab_n + ab_n -ab\right| \\
							  &\le \left|b_n\right|\left|a_n - a\right| + \left|a\right|\left|b_n - b\right|
			\end{align}
			Then use the convergence of $a_n$ and the boundedness of $b_n$ to get
			\begin{align}
				&\le M \left|a_n - a\right| + \left|a\right|\left|b_n - b\right| \\
				&\le M \frac{\e}{2 M} + \left|a\right|\frac{\e}{2 \left|a\right|} = \e
			\end{align}
		\item The final statement will follow from (c) if we can prove that
			\begin{align}
				\left( b_n \right) \to b \implies \left( \frac{1}{b_n} \right) \to \frac{1}{b}
			\end{align}
			Consider
			\begin{align}
				\left|\frac{1}{b_n} - \frac{1}{b}\right| = \frac{\left|b - b_n\right|}{\left|b\right|\left|b_n\right|}
			\end{align}
			Notice that $\exists N_1 \in \N$ such that if $n \ge N_1$ we have $\left|b_n - b\right| < \e$. Before we can continue with our usual strategy, notice that we still have a sequence, $b_n$, in the denominator. We need to find a number that is \textit{smaller} than every element of the sequence so that we can we a fraction that is always \textit{bigger} than $\frac{\left|b_n -b \right|}{\left|b\right|\left|b_n\right|}$. To do this, we will use the fact that $\forall n \in \N$ $\left|b_n\right| > \frac{\left|b\right|}{2}$ which we used in our proof of the Partial Sum Gap Theorem. Choose $N_2$ so that $n \ge N_2$ implies
			\begin{align}
				\left|b_n - b\right| < \frac{\e \left|b\right|^{2}}{2}
			\end{align}
			Now let $N = \text{ max}\{N_1, N_2\} $, the $n \ge N$ implies
			\begin{align}
				\left|\frac{1}{b_n} - \frac{1}{b}\right| = \left|b - b_n\right|\frac{1}{\left|b\right|\left|b_n\right|} < \frac{\e \left|b\right|^{2}}{2} \frac{1}{\left|b\right|\frac{\left|b\right|}{2}} = \e
			\end{align}
	\end{alphabetize}	
\end{proof}


`\begin{theorem}[Order Limit Theorem]
        Let $a,b \in \R$ and $\lim_{} a_n = a$ and $\lim_{} b_n = b$.
        \begin{enumerate}
                \item If $a_n \ge 0$ $\forall n \in \N$, then $a \ge 0$
                \item If $a_n \le b_n$ $\forall n \in \N$, then $a \le b$
                \item If $\exists c \in \R$ such that $c \le b_n$ $\forall n \in \N$, then $c \le b$. Similarly, if $a_n \le c$ $\forall n \in \N$, then $a \le c$.
        \end{enumerate}
\end{theorem}

\begin{proof}
	We will show each part in turn. Notice that parts (b) and (c) can be bootstrapped from part (a).
	\begin{alphabetize}
		\item By contradiction, assume that $a < 0 $, therefore $\exists N \in \N$ such that $\forall n \ge  N$ we have
	        	\begin{align}
       	        		\left|a_n - a\right| < \frac{\left|a\right|}{2} \implies a_n - a < \frac{\left|a\right|}{2} \\
                		\implies a_n < a + \frac{\left|a\right|}{2} < 0 \\
                		\implies a_n < 0 \text{ }.
        		\end{align}
        		A contradiction!
		\item The Algebraic Limit THeorem ensures that the sequence $\left( b_n - a_n \right) $ converges to $b - a$. Because $b_n - a_n \ge 0$, we can apply part (a) to get that $b - a \ge 0$.
		\item Take $a_n = c$ (or $b_n = c$) for all $n \in \N$, and apply (b).	
	\end{alphabetize}
\end{proof}

\begin{definition}[Increasing/Decreasing/Monotone Properties of Sequences]
	A sequence $\left( a_n \right) $ is \textbf{increasing} if $a_n \le a_{n+1}$ for all $n \in \N$ and \textbf{decreasing} if $a_n \ge a_{n+1}$ for all $n \in \N$. A sequence is \textbf{monotone} if it is either increasing or decreasing.
\end{definition}

\begin{theorem}[Monotone Convergence Theorem]
	If a sequence is monotone and bounded, then it converges.
\end{theorem}

\begin{proof}
	Assume that $\left( a_n \right) $ is a bounded, monotone sequence. Then consider the set $\{a_n | n \in \N\} $. We will proceede by using the definition of convergence, so we will need to guess the exact value so that we can use that definition. Since the sequence is bounded, we can let $s = \sup \{a_n | n \in \N\} $ and be sure that $s$ exists. To see why $s$ is a reasonable choice for a value of the limit, consider $s - \e < s$. Since $s$ is the least upper bound of the set, $s - \e$ is not an upper bound and $\exists N \in \N$ such that if $n \ge N$
	\begin{align}
		s - \e < a_N < &a_n \le  s < s + \e \\
		-\e < &a_n - s < \e\\
		\left|a_n - s\right|< \e
	\end{align}
	Note that we needed to know that the sequence is monotonically increasing to write $\forall n > N \left( a_N < a_n \right) $. Hence proved.
\end{proof}

\begin{definition}[Series and the Convergence Property of a Series]
	Let $\left( b_n \right) $ be a sequence. An \textbf{infinite series} if a formal expression of the form
	\begin{align}
		\sum_{n=1}^{\inf } b_n = b_1 + b_2 + b_3 + b_4 + b_5 + \ldots
	\end{align}
	We define the corresponding \textbf{sequence of partial sums} $s_m$ by
	\begin{align}
		s_m = b_1 + b_2 + b_3 + \ldots + b_m,
	\end{align}
	and say that the series $\sum_{n=1}^{\inf } b_n$ \textbf{converges to} $B$ if the sequence $\left( s_m \right) $ converges to $B$. In this case, we write $\sum_{n=1}^{\inf } b_n = B$. 
\end{definition}

\begin{definition}[Subsequence]
	Let $\left( a_n \right) $ be a sequence of real numbers, and let $n_{1} < n_2 < n_3 < n_4 < n_5 < \ldots$ be an increasing sequence of natural numbers. Then the sequence 
	\begin{align}
		\left( a_{n_{1}}, a_{n_{2}}, a_{n_{3}}, a_{n_{4}}, a_{n_{5}}, \ldots \right) 
	\end{align}
	is called a \textbf{subsequence} of $\left( a_n \right) $ and is be denoted by $\left( a_{n_{k}} \right) $, were $k \in  \N$ indexes the sequence.
\end{definition}

\begin{theorem}[Same Limit Theorem of Subsequences]
	Subsequences of a convergent sequence converge to the same limit as the original sequence.
\end{theorem}

\begin{proof}
	Assume that $\left( a_n \right) $ converges to $a$. Then we know that for any $\e > 0$, there exists an $N \in \N$ such that if $n \ge N$ we have
	\begin{align}
		\left|a_n - a\right| < \e
	\end{align}
	Now consider this statement which contains a subsequence of  $\left( a_n \right) $ :
	\begin{align}
		\left|a_{n_{k}} - a\right|
	\end{align}
	We already proved that the we can pick an $n$ such that when $n \ge N$ the former statement is true. Notice that  $n_k > k$ for all $k \in \N$ since the point of $n_k$ is to skip values in the sequence. Therefore, for the ladder statement, all we have to do is pick $k \ge N$ and we are guaranteed to have $n_k > N$ and
		\begin{align}
			\left|a_{n_{k}} - a\right| < \e
		\end{align}
	Therefore $\left( a_{n_{k}} \right) $ converges to $a$ as well.
\end{proof}

\begin{theorem}[Bolzano-Weirstrass Theorem]
	Every bounded sequence contains a convergent subsequence. 	
\end{theorem}

\begin{proof}
	To prove the theorem, we will use the definition of convergence. To use that definition, we will need to come up with an accurate guess for the limit. We will use the Nester Interval Property to construct such a number.

	Assume that $\left( a_n \right) $ is a bounded sequence. Then we know that for all $n \in \N$ we have $\left|a_n\right| \le  M$. Now consider the intervals  $[-M, 0]$ and  $[0, M]$. Because of boundedness, either one or both of these intervals contain an infinite number of points in the sequence. Let $I_1$ be one of the intervals which contains an infinite number of terms and pick some $a_{n_1} \in I_1$ from the sequence to begin constructing a subsequence of $\left( a_n \right) $. Let $I_2$ be the interval obtained by bisecting $I_1$ and then picking an interval with infinitely many terms, and then pick $a_{n_2}$ from the initial sequence so that $a_{n_2} \in I_2$ and $n_2 > n_1$. 

We can continue this process indefinitely by constructing $I_k$ by bisecting $I_{k-1}$, picking a half which contains an infinite number of terms, and selecting $a_{n_k}$ with $a_{n_k} \in I_k$ and $n_k > n_{k-1} > n_{k - 2} > \ldots > n_2 > n_1$ 

Notice that the sets
\begin{align}
	I_1 \supset I_2 \supset I_3 \supset \ldots
\end{align}
form a nested sequence of closed intervals; therefore, by the Nested Interval Property, there exists at least one point $x \in \R$ such that $x \in I_k$. We will now show that $\left( a_n \right) \to x$ which will complete our proof.

Let $\e > 0$. We constructed $I_k$ to have a length of $M\left( 1 / 2 \right)^{k-1}$. From the Algebraic Limit Theorem, we know that this length converges to $0$. Now pick $N$ so that $k \ge N$ implies that the length of $I_k$ is less that $\e$. Because $x$ and $a_{n_k}$ are both in $I_{k}$, and the distance between two points in an interval must be less than or equal to the length of the interval, we have 
\begin{align}
	\left|a_{n_k} - x\right| < \e
\end{align}
\end{proof}

\begin{definition}[Cauchy Property of Sequences]
	A sequence $\left( a_n \right) $ is called a \textbf{Cauchy sequence} if, for every $\e > 0$, there exists an $N \in \N$ such that whenever $m, n \ge N$ if follows that $\left|a_n - a_m\right| < \e$
\end{definition}

\begin{theorem}
	Cauchy sequences are bounded.
\end{theorem}

\begin{proof}
	Let $\left( a_n \right) $ be a Cauchy sequence, then we know for $\e = 1$ if $m, n > N$ we have
	\begin{align}
		\left|a_n - a_m\right| < \e = 1
	\end{align}
	If we set $m = N$, we have
	\begin{align}
		\left|a_n - a_N\right| < 1
	\end{align}
	Now consider $\left|a_n\right|$. We can choose to rewrite this and apply the triangle inequality:
	\begin{align}
		\left|a_n\right| = \left|a_n - a_N + a_N\right| \le \left|a_n - a_N\right| + \left|a_N\right|
	\end{align}
	Using the fact that we have $\left|a_n - a_N\right| < 1$:
	\begin{align}
		\left|a_n\right| < 1 + \left|a_N\right|
	\end{align}
	Finally let
	\begin{align}
		M = \text{ max}\{\left|a_1\right|, \left|a_2\right|, \left|a_3\right|, \ldots, \left|a_{N-1}\right|, \left|a_N\right| + 1\} 
	\end{align}
	Then we know that $\left|a_n\right| < M$ for all $n \in \N$ and the sequence is bounded.
\end{proof}

\begin{theorem}[Cauchy Criterion]
	A sequence converges if and only if it is a Cauchy sequence.
\end{theorem}

\begin{proof}
	We will prove both the forward and backward direction
	\begin{itemize}
		\item ($\implies$) Assume that $\left( x_n \right) $ converges to $L$. Then consider
			\begin{align}
				\left|x_m - x_n\right|\\
				=\left|x_m - L + L - x_n\right|\\
				\le \left|x_m - L\right| + \left|L - x_n\right|\\
				\le \frac{\e}{2} + \frac{\e}{2} = \e
			\end{align}
			Therefore $\left|x_m - x_n\right| \le \e$ and the sequence is Cauchy.
		\item ($\impliedby$) Assume that $\left( x_n \right) $ is a Cauchy sequence, then we know that $\left( x_n \right) $ is bounded. By the Bolzano-Weierstrauss theorem, we know that there exists $\left( x_{n_k} \right) $ which converges. Let
			\begin{align}
				x = \lim_{} x_{n_k}
			\end{align}
			Because $\left( x_n \right) $ is Cauchy, we know that there exists an $N \in \N$ such that if $n, m > N$
			\begin{align}
				\left|x_n - x_m\right| < \frac{\e}{2}
			\end{align}
			By using the definition of convergence and picking $n_K \ge N$, we can also write
			\begin{align}
				\left|x_{n_K} - x\right| < \frac{\e}{2}
			\end{align}
			To see that $N$ has the desired property, observe that if $n \ge N$, then
			\begin{align}
				\left|x_n - x\right| = \left|x_n - x_{n_K} + x_{n_K} - x\right| \\
				\le \left|x_n - x_{n_k}\right| + \left|x_{n_k} - x\right|\\
				< \frac{\e}{2} + \frac{\e}{2} = \e
			\end{align}
	\end{itemize}	
\end{proof}

\begin{theorem}[Algebraic Limit Theorem for Series]
	If $\sum_{n=1}^{\infty} a_n = A$ and $\sum_{n=1}^{\infty} b_n = B$, then
	\begin{romanize}
		\item $\sum_{n=1}^{\infty} c a_n = cA$ for all $c \in \R$ and
		\item $\sum_{n=1}^{\infty} \left( a_n + b_n \right) = A + B$
	\end{romanize}
\end{theorem}

\begin{proof}
	We will prove each part separately.
	\begin{romanize}
		\item We need to show that $\sum_{n=1}^{\infty} ca_n = cA$, and we know that $\sum_{n=1}^{\infty} a_n = A$ so consider the partial sums 
			\begin{align}
				t_m &= c a_1 + c a_2 + c a_3 + \ldots + c a_m \\
				s_m &= a_1 + a_2 + a_3 + \ldots + a_m
			\end{align}
		Notice that $t_m = c s_m$, therefore by the Algebraic Limit Theorem  $\lim_{} t_m = \lim_{} c s_m = cA$. Therefore, since the partial sum converges to $cA$, $\sum_{n=1}^{\infty} c a_n = A$.
		\item Consider the partial sums:
			\begin{align}
				t_m &= (a_1 + b_1) + \left( a_2 + b_2 \right) + \ldots + \left( a_m + b_m \right)\\
				s_m &= a_1 + a_2 + a_3 + \ldots + a_m
			\end{align}
		Notice that $t_m = a_m + b_m$, therefore $\lim_{} t_m = \lim_{} \left( a_m + b_m \right)$, and by the Algebraic Limit Theorem $\lim_{} t_m = A + B$. Therefore, since the partial sum converges to $A + B$, $\sum_{n=1}^{\infty} \left( a_n + b_n \right) = A + B$.

	\end{romanize}
\end{proof}

\begin{theorem}[Cauchy Criterion for Series]
	The series $\sum_{n=1}^{\infty} a_n$ converges if and only if, given $\e > 0$, there exists an $N \in \N$ such that whenever $n > m \ge N$ if follows that
	\begin{align}
		\left|a_{m+1} + a_{m+2} + \ldots + a_n\right| < \e
	\end{align}
\end{theorem}

\begin{proof}
	Notice that 
	\begin{align}
		\left|a_{m+1} + a_{m+2} + \ldots + a_{n}\right|	&= \left|s_n - s_m\right|\\
								&= \left|s_n - L + L - s_m\right|\\
								&\le \left|s_n - L\right| + \left|L - s_m\right|\\
								&\le \frac{\e}{2} + \frac{\e}{2} = \e
	\end{align}
	Hence proved.
\end{proof}

\begin{theorem}
	If the series $\sum_{n=1}^{\infty} a_n$ converges, then $\left( a_n \right) \to 0$.
\end{theorem}

\begin{proof}
	If $\sum_{n=1}^{\infty} a_n$ is a converges, then it is a Cauchy sequence, therefore $\exists N \in \N$ such that if $m, n > N$
	\begin{align}
		\left|a_{m+1} + a_{m+2} + \ldots + a_n\right| < \e
	\end{align}
	Pick $n = m + 1$, then
	\begin{align}
		\left|a_{m+1}\right| < \e
	\end{align}
	Therefore, if $n > m + 1$ 
	\begin{align}
		\left|a_n - 0\right| < \e
	\end{align} 
	Hence proved.
\end{proof}

\begin{theorem}[Comparison Test]
	Assume $\left( a_n \right) $ and $\left( b_n \right) $ are sequences satisfying $0 \le a_n \le  b_n$ for all $n \in \N$.
\end{theorem}

\begin{proof}
	
\end{proof}

\begin{theorem}[Absolute Convergence Test]
	If the series $\sum_{n=1}^{\infty} \left|a_n\right|$ converges, then $\sum_{n=1}^{\infty} a_n$ converges as well. 
\end{theorem}

\begin{proof}
	
\end{proof}

\begin{theorem}[Alternating Series Test]
	Let $\left( a_n \right) $ be a sequence satisfying,
	\begin{romanize}
		\item $a_1 \ge a_2 \ge a_3 \ge  \ldots \ge  a_n \ge  a_{n+1} \ge \ldots$ and
		\item $\left( a_n \right) \to 0$
	\end{romanize}
\end{theorem}

\begin{proof}
	
\end{proof}

\begin{definition}
	If $\sum_{n=1}^{\infty} \left|a_n\right|$ converges, then we say that the original series $\sum_{n=1}^{\infty} a_n$ \textbf{converges absolutely}. If, on the other hand, the series $\sum_{n=1}^{\infty} a_n$ converges but the series of absolute values $\sum_{n=1}^{\infty} \left|a_n\right|$ does not converge, the we say that the original series $\sum_{n=1}^{\infty} a_n$ \textbf{converges conditionally}.
\end{definition}
