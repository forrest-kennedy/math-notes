\lecture{12}{Tuesday 05 February 2025}{02-05-25}

-Sequences and series are the most important part of the class. "If you don't understand this, you are going to fail."

\begin{theorem}
	$\R \setminus \Q$ is uncountable: $\R \setminus \Q \sim \R$
\end{theorem}

\begin{definition}
	Given a set $A$, the power set $P\left( A \right) $ is the set of all subsets of $A$.
\end{definition}

\begin{theorem}
	If $A$ is a finite set with $\left|A\right| = n$ then $\left|P\left( A \right) \right| = 2^{n}$
\end{theorem}

This works even for infinite sets!

\begin{theorem}
	$P\left( \N \right) \sim \R$ 
\end{theorem}

\begin{theorem}
	Given any set $A$, there does not exist a surjective function $f: A \to P\left( A \right) $.
\end{theorem}

This means that if $A$ is infinite, then $P\left( A \right) $ is a "bigger" infinite than $A$.

\begin{eg}
	$\N \to P\left( \N \right) \sim \R \to P\left( P\left( \N \right)  \right) \sim P\left( \R \right) \to \ldots  $
\end{eg}

Won't be asking too many questions about this stuff.

\section{Sequences and Series}

Recall from the example given on day 1 that we cannot sum up infinite stiff. Instead, you add up finitely many things and then take a "limit." Now we define what a limit is.

\begin{definition}
	A sequence is a function whose domain is $\N$.
\end{definition}

\begin{eg}
	$2, 4, 8, 16, 32, \ldots$ is a sequence of natural numbers. $\pi, \pi^{2}, \pi^{3}, \ldots$ is a sequence of real numbers.
\end{eg}

Sequences are not series. Limits apply to sequences, not series.

\begin{eg}
	$\left( 1, \frac{1}{2}, \frac{1}{3}, \ldots \right) $
\end{eg}

\begin{eg}
	$\left( \frac{1+n}{n}_{n=1}^{\inf} \right) = \left( 2, \frac{3}{2}, \frac{4}{3}, \ldots \right) $	
\end{eg}

\begin{eg}
	$\left( \frac{1+n}{n} \right) $ 

	If you do not write the starting and ending points, it is assume that it is $n=1$ to $\inf$.
\end{eg}

\begin{eg}
	$\left( a_n \right) $, where $a_n = 2^{n}$ for all $n \in \N$.
\end{eg}

\begin{eg}
	$\left( x_n \right)$, where $x_1 = 2$ and $\frac{x_n + 1}{2 }$ for all $n \ge 1$.
\end{eg}

\begin{definition}[Convergence of a Sequence]
	A sequence $a_n$ converges to a real number $a$, if for every $\e > 0$, there exists $N \in \N$ such that whenever $n \ge N$, we have $\left|a_n - a\right| < \e$. In this case we write
	\[
		\lim_{n \to \inf} a_n = a \iff \lim a_n = a \iff \left( a_n \right) \to a
	.\] 
\end{definition}

\begin{definition}[Conversgence of a Sequence Topological Definition]
	A sequence $\left( a_n \right) $ converges to $a$, if every $\e$-neighborhood of $a$ contains all but a finite number of the terms of $\left( a_n \right) $.
\end{definition}

\begin{definition}
	Given $a \in \R$ and $\e > 0$, the set
	\[
		V_\e\left( a \right) = \{x \in \R : \left|x - a\right| < \e\} 
	.\] 
	is called the $\e$-neighborhood of $a$
\end{definition}

\begin{eg}
	Prove $\lim\left( \frac{1}{\sqrt{n}} \right) = 0$
\end{eg}

\begin{proof}
	\begin{enumerate}
		\item Challenge: $\e = \frac{1}{2}$ Response: let  $N = 5$. To confirm, notice $n \ge 5 \implies \left|\frac{1}{\sqrt{n}} - 0\right| = \frac{1}{\sqrt{n} } < \frac{1}{2}$
		\item Challenge: $\e = \frac{1}{10}$. Response: let $N = 101$. To confirm check  $n \ge 101 \implies \frac{1}{\sqrt{n}} < \frac{1}{10}$ 
	\end{enumerate}
\end{proof}

\begin{proof}
	WTS: $\lim\left( \frac{1}{\sqrt{n} } \right) = 0$. If $n \ge N$ we want 
	\begin{align}
		\left|\frac{1}{\sqrt{n} } - 0\right| &< \e \\
		\iff \frac{1}{\sqrt{n}} &< \e \\
		\iff \frac{1}{\e^{2}} &< n
	\end{align}
	Choose $N \in \N$ such that $\frac{1}{\e^{2}} < N \le  n $
\end{proof}

\begin{proof}
	Let $\e >0$ be given. Let  $N \in \N$ be such that 
	 \[
		 N > \frac{1}{\e^{2}}
	.\] 
	Let $n \ge N$. Then we observe that
	\[
		n > \frac{1}{\e^{2}} \implies \frac{1}{\sqrt{n} } < \e \implies\left|\frac{1}{\sqrt{n}} - 0\right| < \e	
	.\] 
	Hence, the theorem is proved.
\end{proof}

























