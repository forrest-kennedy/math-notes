\lecture{8}{Wednesday 29 January 2025}{}
- Math Club in MATH350

\begin{definition}
	Let $A, B$ be two sets, we say that $A$ has the \textbf{same cardinality} as $B$ if there exists $f:A\to B $ which is a bijection. In the case we write $A \sim B$. Note that $A\sim B \iff B \sim A$
\end{definition}


\begin{eg}
	$A = \{1, 2\}$, $B= \{apple, bananana\}$. Then $A \sim B$ since we can define $f: A \to B$ such that:
	\[
	f\left( x \right) =
	\begin{cases}
		f\left( 1 \right) & = \text{apple}\\
		f\left( 2 \right) & = \text{banana}
	\end{cases}
	.\] 
	$f$ is a bijection, so $A \sim B$
\end{eg}

\begin{eg}
	let $E = \{2, 4, 6, 8, \ldots\}$. Claim: $\N \sim E$. Define $f: \N \to E$ given by: 
	\[
		\begin{cases}
			f\left( 1 \right) & = 2 \\
			f\left( 2 \right) & = 4 \\
			f\left( 3 \right) & = 6 \\
			\ldots
		\end{cases}
	.\] 
	$f$ is a bijection, so $\N \sim E$
\end{eg}

\begin{eg}
	$\N \sim \Z$
\end{eg}

\begin{proof}
	$f: \N \to \Z$ is given by 
	\[
	f\left( n \right) = 	
	\begin{cases}
		\frac{n-1}{2} & \text{if $n$ is odd}\\
		\frac{-n}{2} & \text{if $n$ is even}.
	
	\end{cases}
	.\] 
	$f$ is a bijection, so $\N \sim \Z$
\end{proof}

\begin{theorem}
	Let $A, B, C$ be sets. If $A \sim B$ and $B \sim C$, then $A \sim C$.
\end{theorem}

\begin{proof}
	As $A \sim B$, hence there exists a bijection $f:A \to B$. As $B \sim C$, there exists a bijection $g:B \to C$. Therefore, $g \circ f: A \to C$ is a bijection
\end{proof}


\begin{theorem}
	Let $X, Y$ be two sets. If there exists an injective function $f: X \to Y$ and an injective function $g: Y \to X$, then there exists a bijection $h: X \to Y$ and hence $X \sim Y$.
\end{theorem}

The above will make our lives easier. We no longer need to find an explicit function. Notice no need to check either function for surjectivity. We get it for free.

\begin{theorem}
	$\N \sim \Z^{2}$ where
	\[
		\Z^{2} = \{\left( m,n \right)  : m, n \in \Z\}
	.\] 
\end{theorem}

\begin{proof}[Informal Proof]
	Take grid of points down to the number line.
\end{proof}

\begin{proof}
	Let $f: \N \to \Z^{2}$ given by 
	\[
		f\left( n \right) = \left( n, 0 \right) 
	.\] 
	$f$ is clearly injective.As $\Z \sim \N$  $\implies$ there exists $g: \Z \to \N$ which is a bijection. Define
	\[
	h: \Z^{2} \to \N	
	.\] 
	where $h\left( m, n \right) = 2^{g\left( m \right) } \cdot 3^{g\left( n \right) }$. Now we will show that $h$ is injective. Assume that $h\left( m_1, n_1 \right) = h\left( m_2, n_2 \right) $. We want to show that $m_1 = m_2$ and $n_1 = n_2$:
	\[
	2^{g\left( m_1 \right) } 3^{g\left( n_1 \right) } = 2^{g\left( n_1 \right)}3^{g\left( n_2 \right) }
	.\] 
	As $2$ and $3$ are prime numbers, by unique factorization:
	\[
	\implies g\left( m_1 \right) = g\left( m_2 \right) \text{ and } g\left( n_1 \right)  = g\left( n_2 \right).
	.\] 

	But $g: \Z \to \N$ is a bijection, hence $m_1 = m_2$ and $n_1 = n_2$ $\implies$ $h$ is injective. Thus, by the Cantor-Schroder-Berstein theorem, there exists $z: \N \to \Z^{2}$ which is a bijection.
\end{proof}


\begin{theorem}
	Show that $\N \to \N^{3}$ where $\N^{3} = \{\left( a, b, c \right) : a, b, c \in \N\}$
\end{theorem}

\begin{proof}
	Let $f: \N \to \N^{2}$ be $f\left( n \right) =  \left( n, 1, 1 \right) $. This $f$ is injective. Let $g: \N^{3} \to \N$ where $g\left( a, b, c \right) = 2^{a}3^{b}5^{c}$. This $g$ is injective by the same logic as before. By CSB, then there exists a bijection $z: \N \to \N^{3}$.
\end{proof}


\begin{theorem}
	A set  $S$ is c called \textbf{countably infinite} if $S \sim \N$. A set $S$ is called \textbf{countable} if either $S$ is finite or countably infinite. $S$ is called \textbf{uncountable} if it is not countable. (This definition's slightly different from the textbook).
\end{theorem}

\begin{eg}
	$A = \{1, 2\}$ is finite and countable. $\Z = \{\ldots, -2, -1, 0, 1, 2, \ldots\}$ is countably infinite and countable.
\end{eg}



