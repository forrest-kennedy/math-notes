\lecture{17}{Friday 14 February 2025}{02-14-25 Lecture}

\begin{eg}
	$a_n = \frac{1}{n^{2} + 10}$ and $\lim_{} a_n = 0$ 
\end{eg}

\begin{align}
	a_n = \frac{1}{n^{2} \left( 1 + \frac{10}{n^{2}} \right) } \\
	\left( \frac{1}{n^{2}} \right) \frac{1}{\left( 1 + \frac{10}{n^{2}} \right) }
\end{align}

We know that $\lim_{} \frac{1}{n} = 0$ so

\begin{align}
	\text{(By ALT) } \lim_{} \frac{1}{n^{2}} = 0 \\
	\lim_{} \left( 1 + \frac{1}{n^{2}} \right) = 1 \\
	\lim_{} \frac{1}{1+\frac{10}{n^{2}}} = 1\\
	\lim_{} \frac{1}{n^{2}} \cdot \frac{1}{\left( 1 + \frac{10}{n^{2}} \right) }= 0.
\end{align}

Hence proved.

\begin{theorem}
	Let $a,b \in \R$ and $\lim_{} a_n = a$ and $\lim_{} b_n = b$.
	\begin{enumerate}
		\item If $a_n \ge 0$ $\forall n \in \N$, then $a \ge 0$ 
		\item If $a_n \le b_n$ $\forall n \in \N$, then $a \le b$ 
		\item If $\exists c \in \R$ such that $c \le b_n$ $\forall n \in \N$, then $c \le b$. Similarly, if $a_n \le c$ $\forall n \in \N$, then $a \le c$.
	\end{enumerate}
\end{theorem}

\begin{proof}
	By contradiction, assume that $a < 0 $, therefore $\exists N \in \N$ such that $\forall n \ge  N$ we have 
	\begin{align}
		\left|a_n - a\right| < \frac{\left|a\right|}{2} \implies a_n - a < \frac{\left|a\right|}{2} \\
		\implies a_n < a + \frac{\left|a\right|}{2} < 0 \\
		\implies a_n < 0 \text{ } \forall n \ge \N.
	\end{align}
	A contradiction!
\end{proof}














