\lecture{5}{Thursday 23 January 2025}{The Definition of Function}


\begin{definition}
	Given sets $A, B$ a function of $A \to B$ is a mapping that takes each element of $A$ to a single element of $B$.
\end{definition}

Note: 

\begin{enumerate}
	\item $f$ is the function, $f\left( x \right)$ not the function.
	\item $A$ is call the \textbf{domain}. $B$ is called the  \textbf{codomain}. $Range\left( f \right) = \{y \in B  \mid  \exists x \in A  \text{ and } f\left( x \right) = y\}$. $Range\left( f \right) \neq$ codomain in general.
\end{enumerate}

\begin{eg}
	Given $f : \R \to \R$ where  $f\left( x \right) = x^{2}$, the comain is $\R$ the codomain is $\R$ and the range is $[0, \inf]$. 
\end{eg}


Note:
\begin{enumerate}
	\item If $f\left( x \right) \neq f\left( y \right)$ when $x \neq  y$ then $f$ is called \textbf{injective} or \textbf{one-to-one}.
	\item If $Range\left( f \right) =$ codomain of $f$ then $f$ is called \textbf{surjective} or \textbf{onto}.
	\item If $f$ is both injective and surjective, then it is called \textbf{bijective}.
\end{enumerate}	

\begin{eg}[Dirichlet Function 1829]
	Define $g: \R \to \R$ by
	\[
		g\left( x \right) =
		\begin{cases}
			0 & \text{if } x \notin  \Q \\
			1 & \text{if } x \in \Q 
		\end{cases}	
	.\]

	The above function definition is important for historical reasons. Dirichlet came up with the definition of a function given above, and it generalizes the concept of a function nicely. Before Dirichlet, function were either thought about as "nice" graphs or as formula, but the new definition generalizes both of these and allows for less traditional function definitions.
\end{eg}

\begin{eg}
	The \textbf{absolute value function}, $ \mid \cdot  \mid : \R \to \R$, is given below: 
	\[
		 \mid x \mid =
		\begin{cases}
			x & \text{if } x\ge 0	\\
			-x & \text{if } x <0
		\end{cases}
	.\] 
\end{eg}

\begin{theorem}
	Given the above definition of the absolute value function, we have:
	\begin{enumerate}
		\item  $ \mid ab \mid  =  \mid a \mid  \mid b \mid $
		\item $ \mid a + b \mid \le  \mid a \mid + \mid b \mid $ \textbf{ (Triangle Inequality)}
	\end{enumerate}
\end{theorem}

\begin{proof}
	
\end{proof}


\begin{note}
	A common trick that we will use in Analysis is the "add/subtract" trick. Let $a, b, c \in \R$, then: 
	\begin{align}
		\lvert a - b \rvert = \lvert \lvert a - c \rvert + \lvert c - b \rvert \rvert\\
		\implies \lvert a-b \rvert \le \lvert \lvert a - c \rvert + \lvert c - b \rvert \rvert
	\end{align}
\end{note}

\begin{theorem}
	Let $a, b \in R$. Then $a = b \iff \lvert a - b \rvert < \epsilon$ for all $\epsilon > 0$.
\end{theorem}

\begin{proof}
	$\left( \implies \right)$ If $a=b$, then $a-b=0$ and $a - b < \epsilon$ for all $\epsilon > 0$. 
	$\left( \impliedby \right) $ FSOC assume $\lvert a-b \rvert < \eps$ for all $\eps > 0 $ and $a \neq b$, then let $\eps_0 = a-b \neq 0$ then we gave $\lvert a-b \rvert < \eps$ and $\lvert a-b \rvert = \eps_0$, a contradiction! 
	
\end{proof}






